\chapter{A \textsc{Patus} Walkthrough Example}
\label{sec:walkthrough}

In this section, we give an example, who a \textsc{Patus} stencil specification
can be derived from a mathematical problem description.
From this stencil specification, the \textsc{Patus} code generator generates a
C code which implements the compute kernel along with a benchmarking harness that
can be used to measure the performance of the generated code and also shows how
the kernel is to be called from within a user code.


\section{From a Model to a Stencil}

Consider the classical wave equation\index{wave equation} on $\Omega=[-1,1]^3$ with Dirichlet boundary conditions and some initial condition:
\begin{align}
  \frac{\partial^2 u}{\partial t^2} - c^2 \Delta u & = 0 \qquad \text{in }\Omega, \nonumber \\
  u & = g \qquad \text{on }\partial\Omega, \\
  u|_{t=0} & = f. \nonumber
  \label{eq:wave}
\end{align}

We use an explicit finite difference method\index{finite difference}\index{PDE} to discretize the equation both in space and time.
For the discretization in time we use a second-order scheme with time step $\delta t$.
For the discretization in space 
we choose a fourth-order discretization of the Laplacian $\Delta$ on the structured uniformly discretized grid
$\Omega_{h}$ with step size $h$. This discretization gives us
\begin{equation}
	\label{eq:wave-discrete}
	\frac{u^{(t+\delta t)}-2u^{(t)}+u^{(t-\delta t)}}{\delta t} - c^2 \Delta_h u^{(t)} = 0,
\end{equation}
where $\Delta_h$ is the discretized version of the Laplacian:
\begin{align}
	\label{eq:wave-discrete-laplacian}
	\Delta_h u^{(t)}(x,y,z) = \tfrac{-15}{2h^2} & u^{(t)}(x,y,z) + \\
                     \tfrac{-1}{12h^2} & \left(u^{(t)}(x-2h,y,z) + u^{(t)}(x,y-2h,z) + u^{(t)}(x,y,z-2h) \right) + \nonumber \\
                     \tfrac{4}{3h^2}   & \left(u^{(t)}(x-h,y,z)  + u^{(t)}(x,y-h,z)  + u^{(t)}(x,y,z-h)  \right) + \nonumber \\                     
                     \tfrac{4}{3h^2}   & \left(u^{(t)}(x+h,y,z)  + u^{(t)}(x,y+h,z)  + u^{(t)}(x,y,z+h)  \right) + \nonumber \\
                     \tfrac{-1}{12h^2} & \left(u^{(t)}(x+2h,y,z) + u^{(t)}(x,y+2h,z) + u^{(t)}(x,y,z+2h) \right). \nonumber
\end{align}
Substituting Eqn. \ref{eq:wave-discrete-laplacian} into Eqn. \ref{eq:wave-discrete},
solving Eqn. \ref{eq:wave-discrete} for $u^{(t+\delta t)}$, and interpreting $u$ as a grid in space and time with mesh size $h$
and time step $\delta t$, we arrive at
\begin{align*}
	u[x,y,z;t+1] =  2 & u[x,y,z;t] - u[x,y,z;t-1] + c^2\tfrac{\delta t}{h^2} \Big( \tfrac{-15}{2} u[x,y,z;t] + \\
	                \tfrac{-1}{12}& \left(u[x-2,y,z;t] + u[x,y-2,z;t] + u[x,y,z-2;t] \right) + \\
                    \tfrac{4}{3}  & \left(u[x-1,y,z;t] + u[x,y-1,z;t] + u[x,y,z-1;t] \right) + \\                     
                    \tfrac{4}{3}  & \left(u[x+1,y,z;t] + u[x,y+1,z;t] + u[x,y,z+1;t] \right) + \\
                    \tfrac{-1}{12}& \left(u[x+2,y,z;t] + u[x,y+2,z;t] + u[x,y,z+2;t] \right) \!\!\Big).
\end{align*}

To actually solve the discretized equation, we need to specify the mesh size $h$ or, equivalently, the number of grid points $N$,
and the time step $\delta t$ or, equivalently, the number of time steps $t_{\text{max}}$.
Choosing a concrete number of the number of time steps, we can transform the above equation together with the
number of grid points and the number of time steps almost trivially into a \textsc{Patus} stencil specification:

\begin{example}{A \textsc{Patus} stencil specification.}
	The listing below shows the \textsc{Patus} stencil specification for the classical wave
	equation discretized by $4^{\text{th}}$ order finite differences in space and by
	$2^{\text{nd}}$ order finite differences in time. Note that the maximum $N$ in the domain
	size specification is inclusive.

\begin{lstlisting}[language=stencil, frame={}]
stencil wave
{
    domainsize = (1 .. N, 1 .. N, 1 .. N);
    t_max = 100;

    operation (float grid u, float param c2dt_h2)
    {
        u[x, y, z; t+1] = 2 * u[x, y, z; t] - u[x, y, z; t-1] +
            c2dt_h2 * (
                -15/2 * u[x, y, z; t] +
                4/3 * (
                    u[x+1, y, z; t] + u[x-1, y, z; t] +
                    u[x, y+1, z; t] + u[x, y-1, z; t] +
                    u[x, y, z+1; t] + u[x, y, z-1; t]
                )
                -1/12 * (
                    u[x+2, y, z; t] + u[x-2, y, z; t] +
                    u[x, y+2, z; t] + u[x, y-2, z; t] +
                    u[x, y, z+2; t] + u[x, y, z-2; t]
                )
            );
    }
}
\end{lstlisting}
\end{example}


\section{Generating The Code}
\index{\textsc{Patus} code generator}

We feed this stencil specification as an input to \textsc{Patus}, which will turn it to C code.
\textsc{Patus} expects two other input files: a template defining how the code will be parallelized
and how code optimizations will be applied, e.g., how loop tiling/cache blocking is applied.
In \textsc{Patus} lingo, this is called a {\em Strategy}. The other input is a description of the
hardware architecture. It defines which code generation back-end to use (e.g., the OpenMP paradigm
for shared memory CPU system, or NVIDIA C for CUDA for NVIDIA GPUs), and how arithmetic operations
and data types are mapped to the corresponding vector intrinsics and vector data types, for instance.

\begin{example}{Generating the C code from a stencil specification.}
	\label{ex:patusgen}
	\footnotesize
	\noindent\texttt{java -jar patus.jar codegen -{}-stencil=wave.stc} \\
	\phantom{XXX}\texttt{-{}-strategy=cacheblocking.stg} \\
	\phantom{XXX}\texttt{-{}-architecture="arch/architectures.xml,Intel x86\_64 SSE"} \\
	\phantom{XXX}\texttt{-{}-outdir=output}
\end{example}

The command in Ex. \ref{ex:patusgen} will create an implementation for the stencil kernel specified
in the file \texttt{wave.stc} and a benchmarking harness for that kernel and file it in the directory
\texttt{output}. The Strategy chosen is a cache blocking strategy defined in the file \texttt{cacheblocking.stg},
which comes with the \textsc{Patus} software. The hardware architecture for which the code will be
generated is defined by the identifier \texttt{Intel x86\_64 SSE}, which can be found in the architecture
definition file, \texttt{arch/architectures.xml}.

After running the \textsc{Patus} code generation, the directory \texttt{output} will contain the following files:
\begin{itemize}
	\item \texttt{kernel.c} --- The implementation of the stencil kernel defined in \texttt{wave.stc}.
	\item \texttt{driver.c} --- The benchmarking harness invoking the stencil kernel and measuring the
		time for the stencil call. It allocates and initializes data with arbitrary values and (by default)
		validates the result returned by the stencil kernel by comparing it to a na\"ive sequential implementation.
	\item \texttt{timer.c} --- Functions related to timing and calculating the performance.
	\item \texttt{patusrt.h} --- Header file for the timing functions.
	\item \texttt{cycle.h} --- Functions for counting clock cycles to do the timing. (This code has been taken from FFTW \cite{fftw05}.)
	\item \texttt{Makefile} --- A GNUmake Makefile to build the benchmarking harness.
\end{itemize}

The benchmarking harness then can be built by typing \texttt{make} on the command line. This will compile and
link the generated code and produce the benchmarking executable, by default called \texttt{bench}.


\section{Running and Tuning}
\index{auto-tuning}

The benchmark executable requires a number of parameters to run:

\begin{example}{Starting the benchmark executable.}
	\footnotesize
	\noindent\texttt{chrmat@palu1:wave> ./bench\\
	\textit{Wrong number of parameters. Syntax:\\
	./bench <N> <cb\_x> <cb\_y> <cb\_z> <chunk> <\_unroll\_p3>}}
\end{example}

\texttt{N} corresponds to the $N$ used in the stencil specification for the definition of the domain size.
If additional identifiers would have been used to define the domain size, they would appear as parameters to
the executable as well. The \texttt{cb\_x}, \texttt{cb\_y}, \texttt{cb\_z}, and \texttt{chunk} arguments come from the
Strategy. They specify the sizes of the cache blocks in $x$, $y$, and $z$ directions and the number of consecutive
cache blocks assigned to one thread.
\textsc{Patus} unrolls the inner-most loop nest containing the stencil evaluation. \texttt{\_unroll\_p3}
selects one of the unroll configuration code variants: by default, \textsc{Patus} creates code variants
with the loop nest unrolled once (i.e., no unrolling is done) or twice in each direction. Since the example
wave stencil is defined in $3$ dimensions, there are $2^3=8$ code variants with different unrolling configurations.

In Ex. \ref{ex:patusrun}, the benchmark executable was run with a domain size of $200^3$ grid points
and an arbitrary cache block size of $16\times 16 \times 16$ grid points per block, one block in a packet per
thread, and the $0^{\text{th}}$ loop unrolling configuration.

\begin{example}{Running the benchmark executable with arbitrary parameters.}
	\label{ex:patusrun}
	\footnotesize
	\noindent\texttt{chrmat@palu1:wave> ./bench 200 16 16 16 1 0\\
	\textit{Flops / stencil call:\phantom{XX}19\\
	Stencil computations:\phantom{XX}40000000\\
	Bytes transferred:\phantom{XXXXX}509379840\\
	Total Flops:\phantom{XXXXXXXXXXX}760000000\\
	Seconds elapsed:\phantom{XXXXXXX}0.230204\\
	Performance:\phantom{XXXXXXXXXXX}3.301418 GFlop/s\\
	Bandwidth utilization:\phantom{X}2.212731 GB/s\\
	506450156.000000\\
	Validation OK.}}
\end{example}

The benchmark executable prints the number of floating point operations per stencil evaluation, the
total number of stencil evaluations that were performed for the time measurement, the number of
transferred bytes and the total number of floating point operations, the time it took to complete
the sweeps and the calculated performance in GFlop/s and bandwidth utilization. The number below that
is a representation of the time spent in the compute kernel, on which the auto-tuner bases the search.
(It tries to minimize that number.) The string ``Validation OK'' says that the validation test (against
the na\"ive, sequential implementation) was passed, i.e., the relative errors did not exceed certain bounds.

The idea is that the user chooses the problem size, $N$, but all the other parameters, which do not change
the problem definition, but rather implementation details which affect the performance, are to be chosen
by the auto-tuner. In the current state of the software, the user still needs some knowledge how to find
the performance-specific parameters (in this case \texttt{cb\_x}, \texttt{cb\_y}, \texttt{cb\_z}, \texttt{chunk},
\texttt{\_unroll\_p3}). An idea for future work is to encapsulate this knowledge in the Strategy, which
defines the parameters, so that an auto-tuner configuration script can be generated in order to fully automate the auto-tuning process.

We choose $N=200$.
Experience tells us that in cases with relatively small domain size (such as $N=200$), a good choice for
\texttt{cb\_x} is $N$. There are several reasons why cutting the domain in the $x$ direction, which is the
unit stride direction (i.e. choosing \texttt{cb\_x} $< N$), is a bad idea. Reading data in a streaming fashion
from DRAM is faster than jumping between addresses. Utilizing full cache lines maximizes data locality.
And the hardware prefetcher is most effective when the constant-stride data streams are as long as possible.
In Ex. \ref{ex:patustune} we let the auto-tuner choose \texttt{cb\_y}, \texttt{cb\_z}, \texttt{chunk},
and \texttt{\_unroll\_p3} by letting \texttt{cb\_y} and \texttt{cb\_z} from $4$ to $N=200$ in increments of $4$
(\texttt{4:4:200}),
and we want the auto-tuner to try all the powers of $2$ between $1$ and $16$ (\texttt{1:*2:16!}) for the fifth command line
argument, \texttt{chunk}, and try all the values between $0$ and $7$ (\texttt{0:7!}) for \texttt{\_unroll\_p3}, by which
all the generated loop unrolling variants are referenced. The exclamation mark specifies that the corresponding
parameter is searched exhaustively, i.e., the auto-tuner is instructed to visit all of the values in the
range specified.

\begin{example}{Running the \textsc{Patus} auto-tuner.}
	\label{ex:patustune}
	\footnotesize
	\noindent\texttt{java -jar patus.jar autotune ./bench 200 200 4:4:200 4:4:200\\
	\phantom{XXX}1:*2:16! 0:7! "C((\textbackslash\$1+\textbackslash\$3-1)/\textbackslash\$3)*((\textbackslash\$1+\textbackslash\$4-1)/\textbackslash\$4)>=\$OMP\_NUM\_THREADS"}
\end{example}

The auto-tuner also allows to add constraints, such as the expression in the last argument in the call in Ex. \ref{ex:patustune}. Constraints are
preceded by a \texttt{C} and can be followed by any comparison expression involving arithmetic.
Parameter values are referenced by a number preceded by a dollar sign \$; the numbering starts with $1$.
In Ex. \ref{ex:patustune}, the condition reads $(\$1+\$3-1)/\$3 \cdot (\$1+\$4-1)/\$4 \geq T$.
The sub-expression $(\$1+\$3-1)/\$3$ is the number of blocks in the $y$-direction,
using the integer division (which rounds towards zero) to express the missing ceiling function\footnote{For positive
integers $a$, $b$ and the integer division $\div$ (defined by $a \div b := \left\lfloor \tfrac{a}{b} \right\rfloor$,
it holds $\left\lceil \tfrac{a}{b} \right\rceil = (a+b-1) \div b$.}). Similarly, $(\$1+\$4-1)/\$4$ is the number of blocks in $z$-direction.
Thus, the condition states that the total number of blocks must be greater or equal than the number of threads
executing the program (assuming the environment variable \texttt{\$OMP\_NUM\_THREADS} is set and controls
how many OpenMP threads the program uses.
Adding constraints is optional, but they can reduce the number of searches by restricting the search space.
In this case, we simply exclude the points in the search space with bad performance, but constraints can be
a helpful tool to suppress invalid configurations. For instance, the numbers of threads per block on a GPU 
must not exceed a particular number lest the program fails. (The numbers are $512$ threads per block on older graphics cards, and $1024$
threads per block on Fermi GPUs).

Ex. \ref{ex:patustuneresult} shows an excerpt of the auto-tuner output.
The program is run for many parameter configurations, and at the end of the search, the auto-tuner displays the parameter
configuration and output of the run for which the best performance was achieved.

\begin{example}{Output of the auto-tuner.}
	\label{ex:patustuneresult}
	\footnotesize
	\noindent\texttt{./bench 200 200 4:4:200 4:4:200 1:*2:16! 0:7!\\
	Parameter set \{ 200 \}\\
	Parameter set \{ 200 \}\\
	Parameter set \{ 4, 8, \dots \}\\
	Parameter set \{ 4, 8, \dots \}\\
	Parameter set , Exhaustive \{ 1, 2, 4, 8, 16 \}\\
	Parameter set , Exhaustive \{ 0, 1, 2, 3, 4, 5, 6, 7 \}\\
	Using optimizer: Powell search method\\
	Executing [./bench, 200, 200, 4, 4, 1, 0]...\\
	\dots\\
	\\
	200 200 36 160 1 3\\
	1.5052755E8\\
	\\
	Program output of the optimal run:\\
	Flops / stencil call:\phantom{XX}19\\
	Stencil computations:\phantom{XX}40000000\\
	Bytes transferred:\phantom{XXXXX}509379840\\
	Total Flops:\phantom{XXXXXXXXXXX}760000000\\
	Seconds elapsed:\phantom{XXXXXXX}0.068421\\
	Performance:\phantom{XXXXXXXXXXX}11.107680 GFlop/s\\
	Bandwidth utilization:\phantom{X}7.444774 GB/s\\
	150527550.000000\\
	Validation OK.}
\end{example}

