\chapter{Integrating into User Code}

By default, \textsc{Patus} creates a C source file named \texttt{kernel.c} 
(The default setting can be overridden with the \texttt{--kernel-file} command line option, cf. Appendix \ref{sec:cmdline}.)
This kernel file contains all the generated code variants of the stencil kernel, a function selecting
one of the code variants, and an initialization function,
\texttt{initialize}, which
does the NUMA-aware data initialization and should preferably be called directly after allocating the data (cf. Chapter \ref{sec:numaawareness}).

\begin{example}{The generated stencil kernel code for the example wave stencil.}
	\label{ex:signatures}
	The generated stencil kernel and the data initialization function have the following signatures:

\begin{lstlisting}[language=C, frame={}]
void wave (float** u_0_1_out,
	float* u_0_m1, float* u_0_0, float* u_0_1, float c2dt_h2,
	int N,
	int cb_x, int cb_y, int cb_z, int chunk, int _unroll_p3);
	
void initialize (
	float* u_0_m1, float* u_0_0, float* u_0_1, float dt_dx_sq,
	int N,
	int cb_x, int cb_y, int cb_z, int chunk);
\end{lstlisting}
\end{example}

The selector function, which is named exactly as the stencil in the stencil specification (\texttt{wave} in Ex. \ref{ex:signatures}),
is the function, which should be called in the user code.
Its parameters are:
\begin{itemize}
	\item pointers to the grid array containing the results; these are double pointers and marked with the \texttt{\_out} suffix;
	\item input grid arrays, one for each time index required to carry out the stencil computation; e.g., in the wave equation
		example, three time indices are required: the result time step $t+1$, which depends on the input time steps $t$ and $t-1$;
		the time index is appended to the grid identifier as last suffix; the \texttt{m} stands for ``minus'';
	\item any parameters defined in the stencil specification, e.g., \texttt{c2dt\_h2} in the wave equation example;
	\item all the variables used to specify the size of the problem domain, only \texttt{N} in our example;
	\item Strategy- and optimization-related parameters, the best values of which were determined by the auto-tuner and
		can be substituted into the stencil kernel function call in the user code;
		the strategy used in the example has one cache block size parameter, (\texttt{cb\_x}, \texttt{cb\_y}, \texttt{cb\_z})
		and a chunk size \texttt{chunk}; furthermore, \texttt{\_unroll\_p3} determines the loop unrolling configuration.
\end{itemize}

The output pointers are required because \textsc{Patus} rotates the time step grids internally,
i.e., the input grids change roles after each spatial sweep. Thus, the user does typically not know which of the grid arrays
contains the solution. A pointer to the solution grid is therefore assigned to the output pointer
upon exit of the kernel function.

The initialization function can be modified to reflect the initial condition required by the problem. However, the generated loop
structure should not be altered. Alternatively, a custom initialization can be done after calling the initialization function
generated by \textsc{Patus}.

