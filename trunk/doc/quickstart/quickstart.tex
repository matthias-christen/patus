\documentclass[a4paper,9pt]{extarticle}

\title{
  {\Huge \textsc{Patus} Quickstart}\\
  {\small \textsc{Patus} v.\patusversion}
}

\author{
  Matthias-M. Christen\\  
  {\small University of Lugano, Switzerland}\\
  {\small \href{mailto:matthias.christen@usi.ch}{matthias.christen@usi.ch}}
}

\pdfinfo{%
  /Title    (Patus Quickstart)
  /Author   (Matthias-M. Christen)
  /Creator  ()
  /Producer ()
  /Subject  ()
  /Keywords (Patus)
}

\newcommand{\patusversion}{0.1}

% *************** Document style definitions ***************

% ******************************************************************
% This file defines the document design.
% Usually it is not necessary to edit this file, but you can change
% the design if you want.
% ******************************************************************

% *************** Load packages ***************

\usepackage{extsizes}

\usepackage[utf8]{inputenc}
\usepackage{avant}
\usepackage{inconsolata}
\renewcommand{\familydefault}{\sfdefault}

\usepackage{graphicx}
\usepackage{epsfig}
\usepackage{amsmath}
\usepackage{amssymb}
\usepackage{amsthm}
\usepackage{url}
\usepackage{listings}
\usepackage{ebnf}
\usepackage[section]{algorithm}
\usepackage{algpseudocode}
\usepackage{mathabx}
\usepackage{multirow}
\usepackage{colortbl}
\usepackage{array}
\usepackage{tikz}
\usepackage{longtable}
\usepackage{framed}
\usepackage{ifpdf}
\usepackage{calc}
\usepackage{soul}


% *************** Add reference to page number at which bibliography entry is cited ***************
\usepackage{citeref}
\renewcommand{\bibitempages}[1]{\newblock {\scriptsize [\mbox{cited at p.\ }#1]}}

% *************** Some colour definitions ***************
\usepackage{color}

\definecolor{greenyellow}   {cmyk}{0.15, 0   , 0.69, 0   }
\definecolor{yellow}        {cmyk}{0   , 0   , 1   , 0   }
\definecolor{goldenrod}     {cmyk}{0   , 0.10, 0.84, 0   }
\definecolor{dandelion}     {cmyk}{0   , 0.29, 0.84, 0   }
\definecolor{apricot}       {cmyk}{0   , 0.32, 0.52, 0   }
\definecolor{peach}         {cmyk}{0   , 0.50, 0.70, 0   }
\definecolor{melon}         {cmyk}{0   , 0.46, 0.50, 0   }
\definecolor{yelloworange}  {cmyk}{0   , 0.42, 1   , 0   }
\definecolor{orange}        {cmyk}{0   , 0.61, 0.87, 0   }
\definecolor{burntorange}   {cmyk}{0   , 0.51, 1   , 0   }
\definecolor{bittersweet}   {cmyk}{0   , 0.75, 1   , 0.24}
\definecolor{redorange}     {cmyk}{0   , 0.77, 0.87, 0   }
\definecolor{mahogany}      {cmyk}{0   , 0.85, 0.87, 0.35}
\definecolor{maroon}        {cmyk}{0   , 0.87, 0.68, 0.32}
\definecolor{brickred}      {cmyk}{0   , 0.89, 0.94, 0.28}
\definecolor{red}           {cmyk}{0   , 1   , 1   , 0   }
\definecolor{orangered}     {cmyk}{0   , 1   , 0.50, 0   }
\definecolor{rubinered}     {cmyk}{0   , 1   , 0.13, 0   }
\definecolor{wildstrawberry}{cmyk}{0   , 0.96, 0.39, 0   }
\definecolor{salmon}        {cmyk}{0   , 0.53, 0.38, 0   }
\definecolor{carnationpink} {cmyk}{0   , 0.63, 0   , 0   }
\definecolor{magenta}       {cmyk}{0   , 1   , 0   , 0   }
\definecolor{violetred}     {cmyk}{0   , 0.81, 0   , 0   }
\definecolor{rhodamine}     {cmyk}{0   , 0.82, 0   , 0   }
\definecolor{mulberry}      {cmyk}{0.34, 0.90, 0   , 0.02}
\definecolor{redviolet}     {cmyk}{0.07, 0.90, 0   , 0.34}
\definecolor{fuchsia}       {cmyk}{0.47, 0.91, 0   , 0.08}
\definecolor{lavender}      {cmyk}{0   , 0.48, 0   , 0   }
\definecolor{thistle}       {cmyk}{0.12, 0.59, 0   , 0   }
\definecolor{orchid}        {cmyk}{0.32, 0.64, 0   , 0   }
\definecolor{darkorchid}    {cmyk}{0.40, 0.80, 0.20, 0   }
\definecolor{purple}        {cmyk}{0.45, 0.86, 0   , 0   }
\definecolor{plum}          {cmyk}{0.50, 1   , 0   , 0   }
\definecolor{violet}        {cmyk}{0.79, 0.88, 0   , 0   }
\definecolor{royalpurple}   {cmyk}{0.75, 0.90, 0   , 0   }
\definecolor{blueviolet}    {cmyk}{0.86, 0.91, 0   , 0.04}
\definecolor{periwinkle}    {cmyk}{0.57, 0.55, 0   , 0   }
\definecolor{cadetblue}     {cmyk}{0.62, 0.57, 0.23, 0   }
\definecolor{cornflowerblue}{cmyk}{0.65, 0.13, 0   , 0   }
\definecolor{midnightblue}  {cmyk}{0.98, 0.13, 0   , 0.43}
\definecolor{darkblue}      {cmyk}{0.92, 0.67, 0   , 0.53}
\definecolor{navyblue}      {cmyk}{0.94, 0.54, 0   , 0   }
\definecolor{royalblue}     {cmyk}{1   , 0.50, 0   , 0   }
\definecolor{blue}          {cmyk}{1   , 1   , 0   , 0   }
\definecolor{cerulean}      {cmyk}{0.94, 0.11, 0   , 0   }
\definecolor{cyan}          {cmyk}{1   , 0   , 0   , 0   }
\definecolor{processblue}   {cmyk}{0.96, 0   , 0   , 0   }
\definecolor{skyblue}       {cmyk}{0.62, 0   , 0.12, 0   }
\definecolor{turquoise}     {cmyk}{0.85, 0   , 0.20, 0   }
\definecolor{tealblue}      {cmyk}{0.86, 0   , 0.34, 0.02}
\definecolor{aquamarine}    {cmyk}{0.82, 0   , 0.30, 0   }
\definecolor{bluegreen}     {cmyk}{0.85, 0   , 0.33, 0   }
\definecolor{emerald}       {cmyk}{1   , 0   , 0.50, 0   }
\definecolor{junglegreen}   {cmyk}{0.99, 0   , 0.52, 0   }
\definecolor{seagreen}      {cmyk}{0.69, 0   , 0.50, 0   }
\definecolor{green}         {cmyk}{1   , 0   , 1   , 0   }
\definecolor{forestgreen}   {cmyk}{0.91, 0   , 0.88, 0.12}
\definecolor{pinegreen}     {cmyk}{0.92, 0   , 0.59, 0.25}
\definecolor{limegreen}     {cmyk}{0.50, 0   , 1   , 0   }
\definecolor{yellowgreen}   {cmyk}{0.44, 0   , 0.74, 0   }
\definecolor{springgreen}   {cmyk}{0.26, 0   , 0.76, 0   }
\definecolor{olivegreen}    {cmyk}{0.64, 0   , 0.95, 0.40}
\definecolor{rawsienna}     {cmyk}{0   , 0.72, 1   , 0.45}
\definecolor{sepia}         {cmyk}{0   , 0.83, 1   , 0.70}
\definecolor{brown}         {cmyk}{0   , 0.81, 1   , 0.60}
\definecolor{tan}           {cmyk}{0.14, 0.42, 0.56, 0   }
\definecolor{gray}          {cmyk}{0   , 0   , 0   , 0.50}
\definecolor{lightgray}     {cmyk}{0   , 0   , 0   , 0.05}
\definecolor{black}         {cmyk}{0   , 0   , 0   , 1   }
\definecolor{white}{cmyk}{0   , 0   , 0   , 0   }%




% *************** Define our own pseudo language for lstlisting ***************
\lstdefinelanguage{pseudo}{
  morekeywords={procedure,function,if,else,for,all,in,to,downto,while,do,end},
  sensitive=true,
  morecomment=[l]{//},
  morecomment=[s]{/*}{*/},
  morestring=[b]"
}

\lstdefinelanguage{pseudo}{
  morekeywords={procedure,function,if,else,for,by,all,in,to,downto,while,do,end},
  sensitive=true,
  morecomment=[l]{//},
  morecomment=[s]{/*}{*/},
  morestring=[b]"
}

\lstdefinelanguage{strategy}{
  keywords={,strategy,stencil,for,by,schedule,subdomain,plane,point,in,parallel,temporary,alias,domain,auto,int,dim,},
  comment=[l]//,
  morecomment=[s]{/*}{*/}
}

\lstdefinelanguage{stencil}{
  keywords={,stencil,operation,boundaries,initial,sum,prod,min,max,grid,double,float,const,param,domainsize,iterate,while,options,},
  comment=[l]//,
  morecomment=[s]{/*}{*/}
}

\lstset{
  language=pseudo,
  basicstyle=\ttfamily,
  frame=ltrb,
  framesep=1pt,
  backgroundcolor=\color{lightgray},
  rulecolor=\color{gray},
  basicstyle=\footnotesize\ttfamily,
  stringstyle=\ttfamily,
  commentstyle=\ttfamily\color{limegreen},
  keywordstyle=\ttfamily\color{midnightblue}\bfseries,
  identifierstyle=\ttfamily,
  tabsize=2,
  showstringspaces=false,
  escapechar=~,
  captionpos=b
}

% *************** Enable hyperlinks in PDF documents ***************
\ifpdf
  \pdfcompresslevel=9
    \usepackage[plainpages=false,pdfpagelabels,bookmarksnumbered,%
      colorlinks=true,%
      linkcolor=darkblue,%
      citecolor=darkblue,%
      filecolor=darkblue,%
      urlcolor=darkblue,%
      pdftex,%
      unicode]{hyperref} 
  \input supp-mis.tex
  \input supp-pdf.tex
  \pdfimageresolution=600
  \usepackage{thumbpdf} 
\else
  \usepackage{hyperref}
\fi

\def\baselinestretch{1.1}

% *************** Algorithms ***************
\renewcommand{\algorithmicrequire}{\textbf{Input:}}
\renewcommand{\algorithmicensure}{\textbf{Output:}}

% *************** Other ***************
\renewcommand{\thefootnote}{\fnsymbol{footnote}}

\newcommand{\boldgreek}[1]{\mbox{\boldmath$#1$}}
\DeclareMathOperator*{\argmin}{arg\,min}

\renewcommand{\_}{\setul{0pt}{.4pt}\ul{$\:\:$}}


% *************** Examples ***************
\newcommand{\example}[1]{
  \vskip 2pt
  \noindent\fcolorbox{gray}{lightgray}{
    \parbox{\textwidth-2\fboxrule-2\fboxsep}{#1}
  }
  \vskip 5pt
}

\newcommand{\cmd}[1]{\\[2pt]
  \texttt{\phantom{XXXX}#1}\\[2pt]
}

\newcommand{\cmdln}[3]{
  %\vspace{8mm}
  \noindent\texttt{--{#1}=}#2
  \begin{quote}#3\end{quote}
}


\newcolumntype{x}[1]{%
>{\centering\arraybackslash}p{#1}}%
% found here: http://texblog.wordpress.com/2008/05/07/fwd-equal-cell-width-right-and-centre-aligned-content/

\newcommand{\w}[1]{\textcolor{white}{\tiny #1}}%

\newcommand\diag[4]{%
  \multicolumn{1}{p{#2}|}{\hskip-\tabcolsep
  $\vcenter{\begin{tikzpicture}[baseline=0,anchor=south west,inner sep=#1]
  \path[use as bounding box] (0,0) rectangle (#2+2\tabcolsep,\baselineskip);
  \node[minimum width={#2+2\tabcolsep},minimum height=\baselineskip+\extrarowheight] (box) {};
  \draw (box.north west) -- (box.south east);
  \node[anchor=south west] at (box.south west) {#3};
  \node[anchor=north east] at (box.north east) {#4};
 \end{tikzpicture}}$\hskip-\tabcolsep}}
% from: http://tex.stackexchange.com/questions/17745/diagonal-lines-in-table-cell

\newcommand\bdiag[4]{%
  \multicolumn{1}{p{#2}|}{\hskip-\tabcolsep
  $\vcenter{\begin{tikzpicture}[baseline=0,anchor=south west,inner sep=#1]
  \path[use as bounding box] (0,0) rectangle (#2+2\tabcolsep,\baselineskip);
  \node[minimum width={#2+2\tabcolsep},minimum height=\baselineskip+\extrarowheight] (box) {};
  \draw (box.north east) -- (box.south west);
  \node[anchor=north west] at (box.north west) {#3};
  \node[anchor=south east] at (box.south east) {#4};
 \end{tikzpicture}}$\hskip-\tabcolsep}}
 

\newcommand{\fixme}[1]{\textbf{\color{red}!!! #1 !!!}}


% *************** End of document style definition ***************


\begin{document}
\maketitle
\date

\section{Introduction}

\textsc{Patus} is a code generation and auto-tuning tool for the class of stencil computations.

Stencil computations are constituent computational building blocks in many
scientific and engineering codes. These codes typically achieve a low
fraction of peak performance. Computational domains that involve stencils
include medical and life science applications, petroleum reservoir
simulations, weather and climate modeling, and physics simulations such as
fluid dynamics or quantum chromodynamics.
Stencil codes may perform tens of thousands of
iterations over the spatial domain in order to resolve the time-dependent
solution accurately; hence, may require significant core hours on
supercomputers. Thus, any performance improvement may lead to a
significant reduction in time to solution.

Despite the apparent simplicity of stencil computations and the fact that their computation structure maps well to 
current hardware architectures, meticulous architecture-specific tuning is still required to elicit a platform's full computational power
due to the fact that microarchitectures have grown increasingly complex.
Manual architecture-specific tuning requires a significant effort:
Not only does it require a deeper understanding of the architecture, but it is also both a time consuming and
error-prone process. Although the performance gain might justify the effort, the code usually becomes nonportable
and hard to maintain.

The goal of \textsc{Patus} is to accept an intuitive stencil specification and turn it into architecture-specific, optimized, high-performance code.
The stencil specification is formulated in a small domain-specific language, which is explained in section \ref{sec:stencilspec}.

Currently, \textsc{Patus} supports shared-memory CPU architectures and, as a proof of concept, CUDA-programmable GPUs.
\chapter*{Copyrights}

ANTLR 3
Copyright (c) 2010 Terence Parr
All rights reserved.

Cetus


JGAP
(LGPL)


Time measurement
Copyright (c) 2003, 2007-8 Matteo Frigo
Copyright (c) 2003, 2007-8 Massachusetts Institute of Technology


\section{Installation}

\subsection{Prerequisites}

\textsc{Patus} works both on Linux and on Microsoft Windows.
It should also work on Mac, but this has not been tested.
\textsc{Patus} assumes that a 64-bit operating system is installed.

There is currently no IDE integration, so the tool has to be started on the command line.
However, in the \texttt{etc} directory of the \textsc{Patus} distribution, you can find syntax files
for \texttt{vi} and \texttt{gedit}.

Prior to using \textsc{Patus}, the following software needs to be installed on your computer:

\begin{itemize}
  \item \textbf{Java 7 or newer.} If you have an older version, please download the JRE 7 or the JDK 7
    from \url{http://www.oracle.com/technetwork/java/javase/downloads/index.html} and follow the
    respective installation instructions.
    
  \item \textbf{Maxima.} Download from \url{http://maxima.sourceforge.net/} and follow the respective
    installation instructions. For Linux distributions, installation packages are available.
    E.g., under Ubuntu, type\\
    \texttt{\phantom{XXXX}sudo apt-get install maxima}\\
    in a shell to install the software.
    
  \item \textbf{gcc.} To compile the code generated by \textsc{Patus}, you will need a recent version
    of the GNU C compiler.\\
    On Linux, if gcc is installed the default version usually should be fine.
    On Windows, we recommend installing \href{http://tdm-gcc.tdragon.net/}{TDM-GCC}, which features
    an easy install of the GCC toolset and MinGW-w64.\\
    Note that the more ``sophisticated'' generated code versions require a 64-bit operating system.
    
  \item \textbf{Make.} To build the generated benchmarking harness and starting the
    auto-tuner, \textsc{Patus} generates Makefiles.\\
    On Linux, if build tools are installed, \texttt{make} will be available.
    On Windows, we recommend installing \href{http://gnuwin32.sourceforge.net/packages/make.htm}{Make for Windows}.
\end{itemize}


\subsection{Installation}

\subsubsection{Linux}

Unzip the \texttt{patus-0.1.zip} file in a directory of your choice.
\cmd{unzip patus-\patusversion.zip}

\noindent Change into the newly created directory.
\cmd{cd Patus}

\noindent Execute the shell script that will set the environment variable
\texttt{PATUS\_HOME} and modify your \texttt{PATH} to include the Patus directory.\\
In bash and sh: \cmd{source util/patusvars.sh}
In csh and tcsh: \cmd{source util/patusvars.csh}
These scripts assume that they are sourced from within the Patus directory.\\[2pt]

\noindent Now you can run \textsc{Patus} by typing
\cmd{patus} on the command line.


\subsubsection{Microsoft Windows}

Microsoft Windows is not supported by \textsc{Patus} as well as Linux, but the \texttt{bin}
directory of \textsc{Patus} contains a batch file which can be used to start \textsc{Patus}.

\medskip
\noindent Unzip the \texttt{patus-0.1.zip} file in a directory of your choice.

\medskip
\noindent \textsc{Patus} can be started by starting \cmd{patus.bat} on the command line.

\section{Using \textsc{Patus}}

\subsection{A Walkthrough Example}
\label{sec:wave}

In \textsc{Patus}, the user specifies the actual stencil computation.
Thus, if the tool is to be used to solve differential equations using a finite difference-based
discretization method, the discretization needs to be done prior to the implementation in \textsc{Patus}.
In the following, we show briefly how this can be done by means of a simple example.


\subsubsection{From a Model to a Stencil Specification}

Consider the classical wave equation\index{wave equation} on $\Omega=[-1,1]^3$ with Dirichlet boundary conditions and some initial condition:
\begin{align}
  \frac{\partial^2 u}{\partial t^2} - c^2 \Delta u & = 0 \qquad \text{in }\Omega, \nonumber \\
  u & = 0 \qquad \text{on }\partial\Omega, \\
  u(x,y,z)|_{t=0} & = \sin(2\pi x) \sin(2\pi y) \sin (2\pi z). \nonumber
  \label{eq:wave}
\end{align}

Using an explicit finite difference method to discretize the equation both in space and time
by means of a fourth-order discretization of the Laplacian $\Delta$ over an equidistant spatial grid with stepsize $h$
and a second-order scheme with time step $\delta t$ in time, we obtain
\begin{equation}
	\label{eq:wave-discrete}
	\frac{u^{(t+\delta t)}-2u^{(t)}+u^{(t-\delta t)}}{\delta t} - c^2 \Delta_h u^{(t)} = 0,
\end{equation}
where $\Delta_h$ is the discretized version of the Laplacian:
\begin{align}
  \label{eq:wave-discrete-laplacian}
  \Delta_h u^{(t)}(x,y,z) = -\tfrac{15}{2h^2} & u^{(t)}(x,y,z) + \\
    -\tfrac{1}{12h^2} & \left(u^{(t)}(x-2h,y,z) + u^{(t)}(x,y-2h,z) + u^{(t)}(x,y,z-2h) \right) + \nonumber \\
     \tfrac{4}{3h^2}  & \left(u^{(t)}(x-h,y,z)  + u^{(t)}(x,y-h,z)  + u^{(t)}(x,y,z-h)  \right) + \nonumber \\                     
     \tfrac{4}{3h^2}  & \left(u^{(t)}(x+h,y,z)  + u^{(t)}(x,y+h,z)  + u^{(t)}(x,y,z+h)  \right) + \nonumber \\
    -\tfrac{1}{12h^2} & \left(u^{(t)}(x+2h,y,z) + u^{(t)}(x,y+2h,z) + u^{(t)}(x,y,z+2h) \right).  \nonumber
\end{align}
Substituting Eqn. \ref{eq:wave-discrete-laplacian} into Eqn. \ref{eq:wave-discrete},
solving Eqn. \ref{eq:wave-discrete} for $u^{(t+\delta t)}$, and interpreting $u$ as a grid in space and time with mesh size $h$
and time step $\delta t$, we arrive at
\begin{align*}
  u[x,y,z;t+1] =  2 & u[x,y,z;t] - u[x,y,z;t-1] + c^2\tfrac{\delta t}{h^2} \Big( -\tfrac{15}{2} u[x,y,z;t] + \\
    -\tfrac{1}{12}& \left(u[x-2,y,z;t] + u[x,y-2,z;t] + u[x,y,z-2;t] \right) + \\
     \tfrac{4}{3}  & \left(u[x-1,y,z;t] + u[x,y-1,z;t] + u[x,y,z-1;t] \right) + \\                     
     \tfrac{4}{3}  & \left(u[x+1,y,z;t] + u[x,y+1,z;t] + u[x,y,z+1;t] \right) + \\
    -\tfrac{1}{12}& \left(u[x+2,y,z;t] + u[x,y+2,z;t] + u[x,y,z+2;t] \right) \!\!\Big).
\end{align*}
This can now be turned into a \textsc{Patus} stencil specification almost trivially:

\begin{lstlisting}[language=stencil]
stencil wave(
  float grid U(0..x_max-1, 0..y_max-1, 0..z_max-1), 
  float param fMin = -1,
  float param fDX = 2 / (x_max-3),
  float param fDT_DX_sq = 0.25)
{
  // do one timestep within the stencil kernel
  iterate while t < 1;
  
  // define the region on which the stencil is evaluated
  domainsize = (2..x_max-3, 2..y_max-3, 2..z_max-3);
  
  // define the initial condition (how the data is initialized
  // before the computation)
  // note that 0 <= x < x_max, etc.
  initial {
    U[x,y,z; -1] = sinf(2*~$\pi$~*((x-1)*fDX+fMin)) *
      sinf(2*~$\pi$~*((y-1)*fDX+fMin)) * sinf(2*~$\pi$~*((z-1)*fDX+fMin));
    U[x, y, z; -1 :
      x==0 || y==0 || z==0 || x==x_max-1 || y==y_max-1 || z==z_max-1 ] = 0;
    U[x,y,z; 0] = U[x,y,z; -1];
    U[x,y,z; 1] = 0;
  }

  // define the actual stencil computation
  operation {
    float c1 = 2 - 15/2 * fDT_DX_sq;
    float c2 = 4/3 * fDT_DX_sq;
    float c3 = -1/12 * fDT_DX_sq;
    
    U[x,y,z; t+1] = c1 * U[x,y,z; t] - U[x,y,z; t-1] +
      c2 * (
        U[x+1,y,z; t] + U[x-1,y,z; t] +
        U[x,y+1,z; t] + U[x,y-1,z; t] +
        U[x,y,z+1; t] + U[x,y,z-1; t]
      ) +
      c3 * (
        U[x+2,y,z; t] + U[x-2,y,z; t] +
        U[x,y+2,z; t] + U[x,y-2,z; t] +
        U[x,y,z+2; t] + U[x,y,z-2; t]
      );
  }
}
\end{lstlisting}

\noindent From this stencil specification, \textsc{Patus} will generate
\begin{itemize}
  \item a C source code file implementing the stencil computation and
  \item source files from which a benchmarking harness can be built.
\end{itemize}
The benchmarking executable is then used by the auto-tuner, which determines a set of
architecture-specific parameters, for which the stencil achieves the best performance.


\subsubsection{The Specification Explained}

\begin{itemize}
  \item The \textbf{stencil} specification defines a stencil ``wave,'' which operates on a grid (called ``U'')
    of size $[$0, x\_max-1$] \times [$0, y\_max-1$] \times [$0,z\_max-1$]$.
    Note that the size parameters do not have to be defined in the stencil specification.
    Instead, they will appear as arguments to the generated stencil kernel function, and, in the
    benchmarking harness, as command line arguments.
    
    The remaining arguments, ``fMin,'' ``fDX,'' and ``fDT\_DX\_sq,'' are used for initializing the grid
    and performing the stencil computation. These parameters will also appear as arguments to the generated
    C function implementing the computation.

    Optionally, parameters can be initialized with default
    values (as shown in the listing above). This affects only the benchmarking harness and will fix the
    values which are passed to the generated stencil kernel.

  \item The \textbf{iterate while} statement defines the number of timesteps to be performed within one stencil kernel call.
    In the example, one timestep per kernel invocation will be performed. (Timesteps are counted from 0.)
    The statement can be omitted. Then, the number of timesteps defaults to 1.
    In the future we will also allow (reduction-based) convergence criteria (e.g., ``iterate as long as the residual is
    larger than some $\varepsilon$'').
    
  \item The \textbf{domainsize} defines the iteration space. While the total size of the grid extends from 0 to $\ast$\_max-1,
    the stencil is only applied to the points between 2 and $\ast$\_max-3 ($\ast \in \{$x, y, z$\}$), i.e., only
    to the \emph{interior} grid points.
    
  \item The \textbf{initial} block defines how the grid is initialized, or more mathematically: it defines the initial condition
    of the (discretized) PDE. The ``sinf'' function is actually a C function (single precision sine)
    which will be called (this is the behavior
    if there is a function which is not known to \textsc{Patus}).
    Note that \textsc{Patus} also defines the $\pi$ literal with the obvious meaning.\\
    The second statement of the initialization sets all the grid points for which x, y, z are 0 or $\ast$\_max-1
    (the condition after colon) to zero. I.e., you can use the set builder notation (with any logical and comparison operators you know from C/C++)
    to select certain grid points and initialize them.\\
    \textbf{initial} blocks are not mandatory; if no \textbf{initial} is provided, \textsc{Patus} will create an initialization
    routine anyway, initializing the data with arbitrary values. (This is to ensure the correct data placement on NUMA machines.)
    
  \item The \textbf{operation}, finally, defines the actual stencil computation.
    It can contain definitions of constants (as in the listing) or temporary values, and it can also contain more than
    one stencil expression.
  
  \item Optionally, the stencil specification can also contain a \textbf{boundaries} block, in which
    special treatment of boundary regions can be specified. Essentially, within the \textbf{boundaries} block,
    special stencils are defined which are applied to boundary regions. See section \ref{sec:stencilspec}.
\end{itemize}


\subsubsection{Building the Benchmarking Harness}

Once the stencil specification is written, \textsc{Patus} can be run to transform it into C code.
Assume the stencil specification was saved in the file \texttt{examples/ stencils/wave-1.stc}.
(This very stencil specification is actually there.)
In your shell (in Linux), type
\cmd{cd examples/stencils\\
%  patus examples/stencils/wave-1.stc --outdir=out/wave-1
  \phantom{XXXX}patus wave-1.stc
}
or, in Microsoft Windows, on the command line type
\cmd{cd examples\textbackslash{}stencils\\
  \phantom{XXXX}..\textbackslash{}..\textbackslash{}bin\textbackslash{}patus.bat wave-1.stc
}
This will generate the C code implementing the Wave stencil, using the default architecture,
and putting the generated files in the
\texttt{out} directory. (You can change the output directly by adding the \texttt{--outdir=$<$your-output-dir$>$}
to the \textsc{Patus} command line.)
Change into the output directory, \cmd{cd out}
and type \cmd{make} to build the benchmarking harness, which will be used for the auto-tuning process, and can also be
used for simple simulations. When the build completes successfully, there is an executable file,
\texttt{bench} (or \texttt{bench.exe} on Windows) in the directory.

If you try to start the benchmarking executable, you will see that it expects some command line arguments:
\example{\small\texttt{%
  \$  ./bench\\
  Wrong number of parameters. Syntax:\\
  ./bench <x\_max> <y\_max> <z\_max> <cb\_x> <cb\_y> <cb\_z> <chunk> <\_unroll\_p3>
}}

The $\ast$\_max correspond to the unspecified domain size variables in the stencil specification.
All the other arguments come from the code generator, and it is the auto-tuner's task to find the best values
for them.


\subsubsection{Auto-Tuning}

The Makefile (which was also used to build the benchmarking harness) defines a ``tune'' target, which
starts the auto-tuner. ``tune'' expects the domain size variables to be specified:
\cmd{make tune x\_max=64 y\_max=64 z\_max=64}
Type the above command in the shell. The auto-tuner will run the benchmarking executable a number of times,
varying the values of the arguments to determine, and it will terminate with a message like

\example{\small
  \texttt{Optimal parameter configuration found:}\\
  \texttt{64 64 64 62 20 8 2 0}\\

  \texttt{Timing information for the optimal run:}\\
  \texttt{0.9047935972598529}\\

  \texttt{Program output of the optimal run:}\\
  \texttt{Flops / stencil call:\phantom{XX}16}\\
  \texttt{Stencil computations:\phantom{XX}1116000}\\
  \texttt{Bytes transferred:\phantom{XXXXX}15728640}\\
  \texttt{Total Flops:\phantom{XXXXXXXXXXX}17856000}\\
  \texttt{Seconds elapsed:\phantom{XXXXXXX}0.000423}\\
  \texttt{Performance:\phantom{XXXXXXXXXXX}42.193517 GFlop/s}\\
  \texttt{Bandwidth utilization:\phantom{X}37.166590 GB/s}\\
  \texttt{1352692.000000}\\
  \texttt{Validation OK.}
}

\noindent Thus, according to the auto-tuner, on the machine the benchmark was run and for x\_max=y\_max=z\_max=64,
the best parameter combination is \cmd{cb\_x=62, cb\_y=20, cb\_z=8, chunk=2, \_unroll\_p3=0.}

\textsc{Patus} has certain requirements for the grid sizes in the unit stride dimension.
If the requirements are not met when you run the executable or the auto-tuner, you may see an error message like
\cmd{Non-native, aligned SIMD type mode requires that (x\_max+2) is divisible by 4 [(x\_max+2) = 102].}
If this is the case, adjust the grid size (here, e.g., set \texttt{x\_max} to 102 instead of 100) and run again.


\subsubsection{Simple Visualization}

You can run the benchmarking executable with the arguments provided by the auto-tuner,
\cmd{./bench 64 64 64 62 20 8 2 0}
If you add the flag \texttt{-o} to the command line,
\cmd{./bench 64 64 64 62 20 8 2 0 -o}
the program will take snapshots of the data grids before and after the execution of the stencil kernel
and write them to text files. If you have gnuplot installed on your system, you can run
\cmd{make plot} which will use gnuplot to create image files from the data text files.


\subsection{Writing Your Own Stencil Specifications}
\label{sec:stencilspec}

A stencil specification has the following form:
\example{
  \texttt{stencil} \textit{\footnotesize stencil-name} \texttt{(} \textit{\footnotesize stencil-arguments} \texttt{)}
  \texttt{\{}\\
  \phantom{XXXX}\textit{\footnotesize iteration-space}\\
  \phantom{XXXX}\textit{[ {\footnotesize number-of-timesteps} ]}\\
  \phantom{XXXX}\textit{\footnotesize operation}\\
  \phantom{XXXX}\textit{[ {\footnotesize boundaries} ]}\\
  \phantom{XXXX}\textit{[ {\footnotesize initial} ]}\\
  \texttt{\}}
}

The order of the iteration space, number of timesteps, operation, boundaries, and initial specifications
is not relevant.


\subsubsection{Stencil Arguments}

Arguments to the stencil can be either \textbf{grid}s or \textbf{param}s. Two datatypes are supported:
\textbf{float} (for single precision floating point numbers) and \textbf{double} (double precision
floating point numbers). Grids are multi-dimensional arrays of data which can be read and written to.
Params are application-specific read-only data used in the computation. They can be scalars or fixed-size
arrays.

\bigskip

\noindent A \textbf{grid} can be declared as follows:
\example{
  \textit{[} \texttt{const} \textit{]}
  \textit{(} \texttt{float} $|$ \texttt{double} \textit{)}
  \texttt{grid}
  \textit{\footnotesize grid-name}
  \textit{[}\texttt{(}
  \textit{\footnotesize lbnd$_1$} \texttt{..} \textit{\footnotesize ubnd$_1$}\texttt{,}
  \textit{\footnotesize lbnd$_2$} \texttt{..} \textit{\footnotesize ubnd$_2$}
  \dots
  \texttt{)}\textit{]}\\
  \textit{[}
  \texttt{[}
  \textit{\footnotesize albnd$_1$} \textit{[}\texttt{..} \textit{\footnotesize aubnd$_1$}\textit{]} \texttt{,}
  \textit{\footnotesize albnd$_2$} \textit{[}\texttt{..} \textit{\footnotesize aubnd$_2$}\textit{]}
  \dots
  \texttt{]}
  \textit{]}
}

\noindent All the upper bounds are inclusive.

\medskip
\noindent Examples:
\begin{lstlisting}[language=stencil]
float grid A
const double grid B
float grid C(-1 .. size_x)
const double grid D(0..sx, 1..sy, -2..sz+2)[3]
float grid E(min_x..max_x, min_y..max_y)[0..1, -1..3]
\end{lstlisting}

The grid \texttt{A} is declared as a single precision grid with no explicit size declaration.
With the size declaration the size of the grid in memory can be defined, which is distinct from the
iteration domain, i.e., the specification of the domain to which the calculation is applied.
In this case, the grid's size is implied from the \textbf{domainsize} in the stencil specification:
the grid is assumed to be the smallest grid such that all the grid accesses in the stencil computation are valid.

For the grid \texttt{B} also no explicit size is declared. \textbf{const} means that the grid is read-only.

Grid \texttt{C} is declared as a one-dimensional grid indexable from -1 to size\_x.
In the actual stencil computation, not all of these points need to be written to (or read), but
no bounds checking is performed. If the computation violates the bounds, the compiled code probably will raise
segmentation faults.

\texttt{D} is declared as an array of 3 three-dimensional grids,
\texttt{E} as a two-di\-men\-sional array of two-dimensional grids.
The examples also show that the lower and upper bounds can be arbitrary arithmetic expressions containing variables.
Any variables appearing in size declarations will become arguments to the generated stencil function.
The array indices of \texttt{E} can be 0 or 1 (first component) and between (and including) -1 and 3 (second component).

\bigskip

\noindent A \texttt{param} can be declared as follows:
\example{
  \textit{(} \texttt{float} $|$ \texttt{double} \textit{)}
  \texttt{param}
  \textit{\footnotesize param-name}
  \textit{[}
  \texttt{[}
  \textit{\footnotesize albnd$_1$} \textit{[}\texttt{..} \textit{\footnotesize aubnd$_1$}\textit{]} \texttt{,}
  \textit{\footnotesize albnd$_2$} \textit{[}\texttt{..} \textit{\footnotesize aubnd$_2$}\textit{]}
  \dots
  \texttt{]}
  \textit{]}
  \textit{[}\texttt{=} \textit{\footnotesize values}
  \textit{]}
}

\medskip
\noindent Examples:
\begin{lstlisting}[language=stencil]
float param a
double param b[2]
float param c = 1.23
double param d[2..5, 3] = {{0.1,0.2,0.3,0.4}, {1.1,2.2,3.3,4.4}, {5,6,7,8}}
\end{lstlisting}

\textbf{param}s can be simple scalar values or, as \textbf{grid}s, (multi-dimensional) arrays of fixed size.
Furthermore, default values can be provided. If the \textbf{param} is an array, the default values are grouped
by curly braces (cf. the initialization of \texttt{d}).


\subsubsection{Iteration Space}

The iteration space, i.e., the domain of the grids on which the stencil computation is executed, is defined by
\example{
  \texttt{domainsize = (}
  \textit{\footnotesize lbnd$_1$} \texttt{..} \textit{\footnotesize ubnd$_1$} \texttt{,}
  \textit{\footnotesize lbnd$_2$} \texttt{..} \textit{\footnotesize ubnd$_2$} \dots
  \texttt{);}
}

\noindent Again, the upper bounds are inclusive. The domain size definition also defines the dimensionality of the stencil.


\subsubsection{Number of Timesteps}

The number of timesteps to be performed within one call to the generated stencil kernel function is defined by
\example{
  \texttt{iterate while} \textit{\footnotesize condition} \texttt{;}
}

\noindent Currently, the \textit{condition} must have the form ``\texttt{t <} \textit{num-timesteps}'' where
\textit{num-timesteps} is an integer literal.
The statement is optional. If it is not specified, by default, one timestep will be performed (this corresponds to
``\texttt{iterate while t < 1}'').

In the future, reduction-based stopping criteria will also be supported, which, e.g., can check for convergence.


\subsubsection{The ``operation''}

In the \textbf{operation} block, the actual stencil computation is specified which is executed on the inner
points of the grid. An operation can contain one or more statements within curly braces.
Statements can be either computations of temporary values or assignments to read-and-write grids
defined in the stencil's parameter list.
Any of arithmetic operators (\texttt{+}, \texttt{-}, $\ast$, \texttt{/}, \texttt{\textasciicircum}) can be used
(for addition, subtraction, multiplication, division, and exponentiation).

Grids are indexed as follows:
\example{
  \textit{\footnotesize grid-name} \texttt{[}
  \texttt{x} \textit{[ (} \texttt{+}$|$\texttt{-} \textit{) \textit{\footnotesize offset$_x$} ]} \texttt{,}
  \texttt{y} \textit{[ (} \texttt{+}$|$\texttt{-} \textit{) \textit{\footnotesize offset$_y$} ]} \texttt{,}
  \dots\\
  \textit{[} \texttt{;}
  \texttt{t} \textit{[ (} \texttt{+}$|$\texttt{-} \textit{) \textit{\footnotesize time-offset} ]}
  \textit{]}\\
  \textit{[} \texttt{;} \textit{\footnotesize idx$_1$} \texttt{,} \textit{\footnotesize idx$_2$} \dots  \textit{]}  
  \texttt{]}
}
\noindent i.e., by spatial coordinates first, then by the temporal index (which must not be present if the grid
was declared as \texttt{const}), and then by array indices if the grid was declared to be an array.

The offsets to the spatial identifiers (\texttt{x}, \texttt{y}, \texttt{z}, \texttt{u}, \texttt{v}, \texttt{w},
\texttt{x}\textit{i} for some integer \textit{i}) and to the spatial identifier \texttt{t} have to be compile-time constants.
The array indices also have to be compile-time constants (e.g., integer literals).

\medskip
\noindent Examples for accessing the grids defined above:
\begin{lstlisting}[language=stencil]
A[x,y+1,z-2; t+2]    // assuming that the stencil dimensionality is 3
B[x+1,y,z]           // B is const => no time component
C[x+1; t-1]
D[x,y,z; 0]          // D is const => no time component, but an array index
E[x,y,z; t+1; 0,-1]  // time component and two-dimensional array index
\end{lstlisting}

Furthermore, temporary variables (scalars or arrays) can be declared and assigned expressions.
Assigning values to arrays works as described for stencil arguments.

The stencil \textbf{operation} can also contain reductions (sums or products), which are resolved at compile time.
This feature is syntactic sugar which can save some typing.
The syntax is
\example{
  \texttt{\{}
  \textit{\footnotesize idxvar$_1$}\texttt{=}\textit{\footnotesize lbnd$_1$}\texttt{..}\textit{\footnotesize ubnd$_1$} \texttt{,}
  \textit{\footnotesize idxvar$_2$}\texttt{=}\textit{\footnotesize lbnd$_2$}\texttt{..}\textit{\footnotesize ubnd$_2$} \dots
  \textit{[} \texttt{:} \textit{\footnotesize constraints} \textit{]}
  \texttt{\}}\\
  \textit{(} \texttt{sum} $|$ \texttt{product} \textit{)}
  \texttt{(} \textit{\footnotesize expression} \texttt{)}
}
\noindent The set defines the index space for the sum or the product, and the expression within the sum or the product
the expression, possibly depending on index variables, which is summed up or multiplied for all index variable values.
Optionally, separated by a colon, in the index set definition, constraints can be specified, which are logical expressions
separated by commas.

The following example shows an \textbf{operation} which uses both a temporary array and a sum:

\begin{lstlisting}[language=stencil]
operation {
  float c[0..2] = {
    2 - 15/2 * fDT_DX_sq,
    4/3 * fDT_DX_sq,
    -1/12 * fDT_DX_sq
  };
    
  U[x,y,z; t+1] = c[0]*U[x,y,z; t] - U[x,y,z; t-1] + (
    { i=-1..1, j=-1..1, k=-1..1, r=1..2 : i^2+j^2+k^2==1 } sum(
      c[r] * U[x+r*i,y+r*j,z+r*k; t]
    )
  );
}
\end{lstlisting}
Here, a temporary array \texttt{c} is declared and initialized, and in the second part a compile time reduction is used
to sum up the neighboring points of the center point:
The three index variables \texttt{i}, \texttt{j}, \texttt{k} take the values -1, 0, and 1, but all combinations are filtered
except if \texttt{i}$^2+$\texttt{j}$^2+$\texttt{k}$^2$=1. \texttt{r} takes the values 1 and 2.
Note that the index of \texttt{c} and the spatial offsets in the grid \texttt{U} within the \texttt{sum} are compile-time
constants (the variables appear only in the sum's index set).


\subsubsection{Boundary Conditions}

The \textbf{boundaries} block can be used to specify special stencils which are to be evaluated in boundary regions.
A boundary point is characterized by having one or more spatial coordinates in the grid access fixed to discrete values.
For instance:

\begin{lstlisting}[language=stencil]
boundaries {
  U[0,y,z;t+1] = 0;
  U[x,y,z; t+1 : 0<=x<=1] = 1;
  U[x,0,z;t+1] = U[x,y_max-1,z;t];
}
\end{lstlisting}
In first statement of the example, all the points with 0 x-coordinates will be set to zero.
In the second statement all the points for which the x-coordinate is either 0 or 1 will be set to 1 (the colon notation is the
same as mentioned previously for compile-time reductions). ``\texttt{0<=x<=1}'' is a shorthand for ``\texttt{0<=x \&\& x<=1}''.
The third statement shows that the right-hand side does not have to be a constant, but can be in fact any arithmetic expression.

If the boundary block is omitted, no boundary handling is implemented in the generated stencil function.


\subsubsection{Initial Condition}

In the \textbf{initial} condition specification, the temporal coordinates of grids are fixed (as opposed to the spatial ones in the
boundary specification). All the grids read in the \texttt{operation} should be initialized with 0 substituted for \texttt{t}.
\texttt{initial} is optional; if it is not provided, \textsc{Patus} creates a default initialization with arbitrary values.

Again, as in the \textbf{boundary} block, the set-builder notation can be used to select a number of grid points to initialize.
E.g., in the last line in the example below, all the points within a cylinder around the origin with radius 30 parallel to the z-axis
are selected.
No error is raised if points are selected that are not contained in the actual grid.
The first statement shows that points can be initialized depending on their grid coordinates.

\begin{lstlisting}[language=stencil]
initial {
  U[x,y,z; -1] = sinf(2*~$\pi$~*((x-1)*fDX+fMin)) *
    sinf(2*~$\pi$~*((y-1)*fDX+fMin)) * sinf(2*~$\pi$~*((z-1)*fDX+fMin));
  U[x, y, z; 0] = U[x, y, z; -1];
  U[x, y, z; 1] = 0;

  U[x,y,z;-1 : 10<=x<=20, 5<=y<=10] = 0;
  U[x,y,z; 0 : x^2+y^2 <= 900] = 0;
}
\end{lstlisting}


\subsection{Integrating Into Your Own Application}


In the generated code (usually the file implementing the stencils in C is called \texttt{kernel.c}),
search for the function with the same name as the name of your stencil to determine the signature.
This is the function that should be called from within your application code.

See the directory \texttt{examples/applications} for examples how to integrate \textsc{Patus}-generated code
into user code.
The examples also contain Makefiles that show how the process of generating and automatically tuning stencil kernels
and use them to build the final application executable can be facilitated.
The examples are described briefly below.


%Create Fortran example
\section{Examples}
\label{sec:examples}

In the \textbf{examples} directory some example stencil specifications can be found.

The \textbf{stencil} subdirectory contains single stencil specifications, for which
code can be generated using the provided Makefile. The Makefile builds the benchmarking
harnesses at the same time.

In the \textbf{applications} subdirectory, two example applications can be found, which use
\textsc{Patus} to generate the computate kernels.
For both applications, Makefiles are provided. Typing \cmd{make} in the shell will
build and run the simulations.
In the Makefiles, remove the ``-mavx'' flag from the compiler command line if your CPU does not support AVX.

\begin{itemize}
  \item \textbf{wave} is the simple 3D wave equation solver from \ref{sec:wave}.
    The file \texttt{wave.c} contains both the application code and the embedded
    \textsc{Patus} stencil specification, showing how both parts can be used in a
    single source file.\\
    A simple visualization (using gnuplot) is provided. \cmd{make run} both starts
    the program and the visualization.
    
  \item \textbf{whispering-gallery} is a 2D nano-photonics simulation developed by Max Nolte,
    who has kindly agreed to publish his source code.
    The simulation uses two stencils (defined in the external files \texttt{fdtdE2.stc}
    and \texttt{fdtdH2.stc}) to calculate Maxwell's equation using the FDTD method,
    and a third stencil to integrate the energy density (\texttt{integrate.stc}).\\
    Typing \cmd{make} in the shell will first build the simulation and then start it.
    When the computation completes, the result is visualized with gnuplot and displayed.
\end{itemize}

\chapter{Current Limitations}

In the current state, there are several limitations to the \textsc{Patus} framework:
\begin{itemize}
	\item Only shared memory architectures are supported (specifically: shared memory CPU systems and single-GPU setups).
	\item It is assumed that the evaluation order of the stencils within one spatial sweep is irrelevant.
		Also, always all points within the domain are traversed per sweep. One grid array is read and another
		array is written to. Such a grid traversal is called a Jacobi iteration.
		In particular, this rules out schemes with special traversal rules such as red-black Gauss-Seidel
		iterations.
	\item No extra boundary handling is applied. The stencil is applied to every interior grid point, but not to boundary
		points. I.e., the boundary values are kept constant; this corresponds to Dirichlet boundary conditions.
		To implement boundary conditions, they could be factored into the stencil operation by means of extra coefficient grids.
		In the same way, non-rectilinear domain shapes and non-uniform grids can be emulated by providing the shape and/or
		geometry information encoded as coefficients in additional grids.
		Alternatively, special $(d-1)$-dimensional stencil kernels could be specified for the boundary treatment,
		which are invoked after the $d$-dimensional stencil kernel operating on the interior. This approach, however,
		will invalidate temporal blocking schemes.
	\item The index calculation assumes that the stencil computation is carried out on a flat grid (or a grid which is
		homotopic to a flat grid). In particular, currently no spherical or torical geometries are implemented, which require
		modulo index calculations.
	\item There is no support for temporally blocked schemes yet.
%	\item Fortran limitations: only one timestep, no output pointer
%GPU limitations: only one timestep, no global sync from within kernel (global barriers cannot be programmed as blocks are
%scheduled sequentially to the multiprocessors; one block might wait being scheduled until another block completes;
%if they wait on each other the code deadlocks)
\end{itemize}


%% ********** Appendix 1 **********
\chapter{\textsc{Patus} Grammars}
\label{sec:appendix_grammars}

\section{Stencil DSL Grammar}

In the following, the EBNF grammar for the \textsc{Patus} stencil specifications syntax is given.
The grayed out identifiers have not yet been specified or implemented and will be added eventually
in the future.

\begin{EBNF}
	\item[Stencil]
%		`stencil' \<Identifier> `\{' [ \<Options> ] \<DomainSize> \<NumIterations> \<Operation>
		`stencil' \<Identifier> `\{' [ \<Options> ] \<DomainSize> \<NumIterations> \<Operation>
			\textcolor{gray}{\<Boundary> [ \<Filter> ] [ \<StoppingCriterion> ]} `\}'
	\item[DomainSize]
		`domainsize' `=' \<Box> `;'
	\item[NumIterations]
		`t\_max' `=' \<IntegerExpr> `;'
	\item[Operation]
    	`operation' \<Identifier> `(' \<ParamList> `)' `\{' \{ \<Statement> \} `\}'
	\item[ParamList]
		\{ \<GridDecl> | \<ParamDecl> \}
	\item[Statement]
		\<LHS> `=' \<StencilExpr> `;'
	\item[LHS]
		\<StencilNode> | \<VarDecl>
	\item[StencilExpr]
		\<StencilNode> | \<Identifier> | \<NumberLiteral> | \<FunctionCall> | ( \<UnaryOperator> \<StencilExpr> ) | ( \<StencilExpr> \<BinaryOperator> \<StencilExpr> ) | `(' \<StencilExpr> `)'
	\item[StencilNode]
		\<Identifier> `[' \<SpatialCoords> [ `;' \<TemporalCoord> ] [ `;' \<ArrayIndices> ] `]'
	\item[SpatialCoords]
		(`x' | `y' | `z' | `u' | `v' | `w' | `x' \<IntegerLiteral> ) [ \<Offset> ]
	\item[TemporalCoord]
		`t' [ \<Offset>
	\item[ArrayIndices]
		\<IntegerLiteral> \{ `,' \<IntegerLiteral> \}
	\item[Offset]
		\<UnaryOperator> \<IntegerLiteral>
	\item[FunctionCall]
		\<Identifier> `(' [ \<StencilExpr> \{ `,' \<StencilExpr> \} ] `)'
	\item[IntegerExpr]
		\<Identifier> | \<IntegerLiteral> | \<FunctionCall> | ( \<UnaryOperator> \<IntegerExpr> ) | ( \<IntegerExpr> \<BinaryOperator> \<IntegerExpr> ) | `(' \<IntegerExpr> `)'
	\item[VarDecl]
		\<Type> \<Identifier>
	\item[Box]
		`(' \<Range> \{ `,' \<Range> `)'
	\item[Range]
		\<IntegerExpr> `..' \<IntegerExpr>
	\item[GridDecl]
		[ \<Specifier> ] \<Type> `grid' \<Identifier> [ `(' \<Box> `)' ] [ \<ArrayDecl> ]
	\item[ParamDecl]
		\<Type> `param' \<Identifier> [ \<ArrayDecl> ]
	\item[ArrayDecl]
		`[' \<IntegerLiteral> \{ `,' \<IntegerLiteral> \} `]'
	\item[Specifier]
		`const'
	\item[Type]
		`float' | `double'
	\item[UnaryOperator]
		`+' | `-'
	\item[BinaryOperator]
		`+' | `-' | `*' | `/' | `\^'
%	\item[Options]
%		`options' `\{' \{ \<OptionIdentifier> `=' \<OptionValue> \} `\}'
\end{EBNF}



\section{Strategy DSL Grammar}

The following EBNF grammar specifies the \textsc{Patus} Strategy syntax.
Again, as the project matures, the specification might change so that yet missing aspects of parallelization and
optimization methods can be specified as \textsc{Patus} Strategies.

\begin{EBNF}
	\item[Strategy]
		`strategy' \<Identifier> `(' \<ParamList> `)' \<CompoundStatement>
	\item[ParamList]
		\<SubdomainParam> \{ `,' \<AutoTunerParam> \}
	\item[SubdomainParam]
		`domain' \<Identifier>	
	\item[AutoTunerParam]
		`auto' \<AutoTunerDeclSpec> \<Identifier>
	\item[AutoTunerDeclSpec]
		`int' | `dim' | ( `codim' `(' \<IntegerLiteral> `)' )

	\item[Statement]
		\<DeclarationStatement> | \<AssignmentStatement> | \<CompoundStatement> | \<IfStatement> | \<Loop> %| \<SubCall>
	\item[DeclarationStatement]
		\<DeclSpec> \<Identifier> `;'
	\item[AssignmentStatement]
		\<LValue> `=' \<Expr> `;'
	\item[CompoundStatement]
		`\{' \{ \<Statement> \} `\}'
	\item[IfStatement]
		`if' `(' \<ConditionExpr> `)' \<Statement> [ `else' \<Statement> ]
	\item[Loop]
		( \<RangeIterator> | \<SubdomainIterator> ) [ `parallel' [ \<IntegerLiteral> ] [ `schedule' \<IntegerLiteral> ] ] \<Statement>

	\item[RangeIterator]
		`for' \<Identifier> `=' \<Expr> `..' \<Expr> [ `by' \<Expr> ]

	\item[SubdomainIterator]
		`for' \<SubdomainIteratorDecl> `in' \<Identifier> `(' \<Range> `;' \<Expr> `)'

	\item[SubdomainIteratorDecl]
		\<PointDecl> | \<PlaneDecl> | \<SubdomainDecl>
	\item[PointDecl]
		`point' \<Identifier>
	\item[PlaneDecl]
		`plane' \<Identifier>
	\item[SubdomainDecl]
		`subdomain' \<Identifier> `(' \<Range> `)'
		
	\item[Range]
		\<Vector> \{ \<UnaryOperator> \<ScaledBorder> \}
	\item[Vector]
		\<Subvector> [ `...' [ `,' \<Subvector> ] ]
	\item[Subvector]
		( `:' \{ `,' \<ScalarList> \} ) | \<DimensionIdentifier> | \<DomainSizeExpr> | \<BracketedVector> | \<ScalarList>
	\item[ScalarList]
		\<ScalarRange> \{ `,' \<ScalarRange> \}
	\item[DimensionIdentifier]
		\<Expr> [ `(' \<Vector> `)' ]
	\item[DomainSizeExpr]
		\<SizeProperty> [ `(' \<Vector> `)' ]
	\item[BracketedVector]
		`(' \<Vector> \{ `,' \<Vector> \} `)'
	\item[ScalarRange]
		\<Expr> [ `..' \<Expr> ]
	\item[SizeProperty]
		( `stencil' | \<Identifier> ) `.' ( `size' | `min' | `max' )
			
	\item[ScaledBorder]
		[ \<Expr> \<MultiplicativeOperator> ] \<Border> [ \<MultiplicativeOperator> \<Expr> ]
	\item[Border]
		\<StencilBoxBorder> | \<LiteralBorder>
	\item[StencilBoxBorder]
		`stencil' `.' `box' [ `(' \<Vector> `)' ]
	\item[LiteralBorder]
		`(' \<Vector> `)' `,' `(' \<Vector> `)'

 	\item[LValue]
		\<GridAccess> | \<Identifier>
	\item[GridAccess]
		\<Identifier> `[' \<SpatialIndex> `;' \<Expr> \{ `;' \<Expr> \} `]'
	\item[SpatialIndex]
		\<Identifier> | \<Range>
		
	\item[Expr]
		\<Identifier> | \<NumberLiteral> | \<FunctionCall> | ( \<UnaryOperator> \<Expr> ) | ( \<Expr> \<BinaryOperator> \<Expr> ) | `(' \<Expr> `)'
	\item[FunctionCall]
		\<Identifier> `(' [ \<Expr> \{ `,' \<Expr> \} ] `)'
	\item[ConditionExpr]
		\<ComparisonExpr> | ( \<ConditionExpr> \<LogicalOperator> \<ConditionExpr> )
	\item[ComparisonExpr]
		\<Expr> \<ComparisonOperator> \<Expr>
    
	\item[UnaryOperator]
		`+' | `-'
	\item[MultiplicativeOperator]
		`*'
	\item[BinaryOperator]
		`+' | `-' | `*' | `/' `\%'
	\item[LogicalOperator]
		`||' | `\&\&'
	\item[ComparisonOperator]
		`$<$' | `$<$=' | `==' | `$>$=' | `$>$' | `!='
\end{EBNF}

% ********** End of appendix **********


\section*{Acknowledgments}
\textsc{Patus} has been developed within the \href{http://www.hp2c.ch/}{High Performance and High Productivity Computing}
initiative of the Swiss National Supercomputer Center (CSCS).

We also want to acknowledge Max Nolte for testing and using \textsc{Patus} for his nano-optics simulation project and
for kindly agreeing to publish his code.

%\bibliographystyle{plain}
%{\small\bibliography{patus-doc}}

\end{document}
