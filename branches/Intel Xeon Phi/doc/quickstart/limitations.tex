\section{Current Limitations}

In the current release, there are the following limitations to \textsc{Patus}:
\begin{itemize}
  \item Only shared memory architectures are supported directly
    (specifically: shared memory CPU systems and single-GPU setups).
    However, if you have MPI code that handles communication and synchronization
    for the distributed case, \textsc{Patus}-generated code can be used as a replacement
    for the per-node stencil computation code.
    
  \item It is assumed that the evaluation order of the stencils within one spatial sweep is irrelevant.
    Also, always all points within the domain are traversed per sweep. One grid array is read and another
    array is written to (Jacobi iteration).
    In particular, this rules out schemes with special traversal rules such as red-black Gauss-Seidel
    iterations.
    
  \item Boundary handling is not yet optimized in the generated code.
    Expect performance drops if a boundary specification is included in the stencil specification.
  
  %\item The index calculation assumes that the stencil computation is carried out on a flat grid (or a grid which is
  %  homotopic to a flat grid). In particular, currently no spherical or torical geometries are implemented, which require
  %  modulo index calculations. 
    
  \item There is no support yet for temporally blocked schemes.

  \item Limitation for Fortran interoperability: only one timestep per stencil kernel is supported.
  
  \item Limitation for (CUDA-capable) GPUs: the generated code does not do any significant performance optimizations yet.
    Also, the stencil kernel can only perform one timestep (since CUDA does not allow global synchronization from within
    a kernel).
\end{itemize}

