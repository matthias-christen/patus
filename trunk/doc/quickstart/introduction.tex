\section{Introduction}

\textsc{Patus} is a code generation and auto-tuning tool for the class of stencil computations.

Stencil computations are constituent computational building blocks in many
scientific and engineering codes. These codes typically achieve a low
fraction of peak performance. Computational domains that involve stencils
include medical and life science applications, petroleum reservoir
simulations, weather and climate modeling, and physics simulations such as
fluid dynamics or quantum chromodynamics.
Stencil codes may perform tens of thousands of
iterations over the spatial domain in order to resolve the time-dependent
solution accurately; hence, may require significant core hours on
supercomputers. Thus, any performance improvement may lead to a
significant reduction in time to solution.

Despite the apparent simplicity of stencil computations and the fact that their computation structure maps well to 
current hardware architectures, meticulous architecture-specific tuning is still required to elicit a platform's full computational power
due to the fact that microarchitectures have grown increasingly complex.
Manual architecture-specific tuning requires a significant effort:
Not only does it require a deeper understanding of the architecture, but it is also both a time consuming and
error-prone process. Although the performance gain might justify the effort, the code usually becomes nonportable
and hard to maintain.

The goal of \textsc{Patus} is to accept an intuitive stencil specification and turn it into architecture-specific, optimized, high-performance code.
The stencil specification is formulated in a small domain-specific language, which is explained in section \ref{sec:stencilspec}.

Currently, \textsc{Patus} supports shared-memory CPU architectures and, as a proof of concept, CUDA-programmable GPUs.