\section{Installation}

\subsection{Prerequisites}

\textsc{Patus} works both on Linux and on Microsoft Windows.
It should also work on Mac, but this has not been tested.
\textsc{Patus} assumes that a 64-bit operating system is installed.

There is currently no IDE integration, so the tool has to be started on the command line.
However, in the \texttt{etc} directory of the \textsc{Patus} distribution, you can find syntax files
for \texttt{vi} and \texttt{gedit}.

Prior to using \textsc{Patus}, the following software needs to be installed on your computer:

\begin{itemize}
  \item \textbf{Java 7 or newer.} If you have an older version, please download the JRE 7 or the JDK 7
    from \url{http://www.oracle.com/technetwork/java/javase/downloads/index.html} and follow the
    respective installation instructions.
    
  \item \textbf{Maxima.} Download from \url{http://maxima.sourceforge.net/} and follow the respective
    installation instructions. For Linux distributions, installation packages are available.
    E.g., under Ubuntu, type\\
    \texttt{\phantom{XXXX}sudo apt-get install maxima}\\
    in a shell to install the software.
    
  \item \textbf{gcc.} To compile the code generated by \textsc{Patus}, you will need a recent version
    of the GNU C compiler.\\
    On Linux, if gcc is installed the default version usually should be fine.
    On Windows, we recommend installing \href{http://tdm-gcc.tdragon.net/}{TDM-GCC}, which features
    an easy install of the GCC toolset and MinGW-w64.\\
    Note that the more ``sophisticated'' generated code versions require a 64-bit operating system.
    
  \item \textbf{Make.} To build the generated benchmarking harness and starting the
    auto-tuner, \textsc{Patus} generates Makefiles.\\
    On Linux, if build tools are installed, \texttt{make} will be available.
    On Windows, we recommend installing \href{http://gnuwin32.sourceforge.net/packages/make.htm}{Make for Windows}.
\end{itemize}


\subsection{Installation}

\subsubsection{Linux}

Unzip the \texttt{patus-0.1.zip} file in a directory of your choice.
\cmd{unzip patus-\patusversion.zip}

\noindent Change into the newly created directory.
\cmd{cd Patus}

\noindent Execute the shell script that will set the environment variable
\texttt{PATUS\_HOME} and modify your \texttt{PATH} to include the Patus directory.\\
In bash and sh: \cmd{source util/patusvars.sh}
In csh and tcsh: \cmd{source util/patusvars.csh}
These scripts assume that they are sourced from within the Patus directory.\\[2pt]

\noindent Now you can run \textsc{Patus} by typing
\cmd{patus} on the command line.


\subsubsection{Microsoft Windows}

Microsoft Windows is not supported by \textsc{Patus} as well as Linux, but the \texttt{bin}
directory of \textsc{Patus} contains a batch file which can be used to start \textsc{Patus}.

\medskip
\noindent Unzip the \texttt{patus-0.1.zip} file in a directory of your choice.

\medskip
\noindent \textsc{Patus} can be started by starting \cmd{patus.bat} on the command line.
