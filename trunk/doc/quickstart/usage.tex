\section{Using \textsc{Patus}}

\subsection{A Walkthrough Example}

In \textsc{Patus}, the user specifies the actual stencil computation.
Thus, if the tool is to be used to solve differential equations using a finite difference-based
discretization method, the discretization needs to be done prior to the implementation in \textsc{Patus}.
In the following, we show briefly how this can be done by means of a simple example.


\subsubsection{From a Model to a Stencil Specification}

Consider the classical wave equation\index{wave equation} on $\Omega=[-1,1]^3$ with Dirichlet boundary conditions and some initial condition:
\begin{align}
  \frac{\partial^2 u}{\partial t^2} - c^2 \Delta u & = 0 \qquad \text{in }\Omega, \nonumber \\
  u & = 0 \qquad \text{on }\partial\Omega, \\
  u(x,y,z)|_{t=0} & = \sin(2\pi x) \sin(2\pi y) \sin (2\pi z). \nonumber
  \label{eq:wave}
\end{align}

Using an explicit finite difference method to discretize the equation both in space and time
by means of a fourth-order discretization of the Laplacian $\Delta$ over an equidistant spatial grid with stepsize $h$
and a second-order scheme with time step $\delta t$ in time, we obtain
\begin{equation}
	\label{eq:wave-discrete}
	\frac{u^{(t+\delta t)}-2u^{(t)}+u^{(t-\delta t)}}{\delta t} - c^2 \Delta_h u^{(t)} = 0,
\end{equation}
where $\Delta_h$ is the discretized version of the Laplacian:
\begin{align}
  \label{eq:wave-discrete-laplacian}
  \Delta_h u^{(t)}(x,y,z) = -\tfrac{15}{2h^2} & u^{(t)}(x,y,z) + \\
    -\tfrac{1}{12h^2} & \left(u^{(t)}(x-2h,y,z) + u^{(t)}(x,y-2h,z) + u^{(t)}(x,y,z-2h) \right) + \nonumber \\
     \tfrac{4}{3h^2}  & \left(u^{(t)}(x-h,y,z)  + u^{(t)}(x,y-h,z)  + u^{(t)}(x,y,z-h)  \right) + \nonumber \\                     
     \tfrac{4}{3h^2}  & \left(u^{(t)}(x+h,y,z)  + u^{(t)}(x,y+h,z)  + u^{(t)}(x,y,z+h)  \right) + \nonumber \\
    -\tfrac{1}{12h^2} & \left(u^{(t)}(x+2h,y,z) + u^{(t)}(x,y+2h,z) + u^{(t)}(x,y,z+2h) \right).  \nonumber
\end{align}
Substituting Eqn. \ref{eq:wave-discrete-laplacian} into Eqn. \ref{eq:wave-discrete},
solving Eqn. \ref{eq:wave-discrete} for $u^{(t+\delta t)}$, and interpreting $u$ as a grid in space and time with mesh size $h$
and time step $\delta t$, we arrive at
\begin{align*}
  u[x,y,z;t+1] =  2 & u[x,y,z;t] - u[x,y,z;t-1] + c^2\tfrac{\delta t}{h^2} \Big( -\tfrac{15}{2} u[x,y,z;t] + \\
    -\tfrac{1}{12}& \left(u[x-2,y,z;t] + u[x,y-2,z;t] + u[x,y,z-2;t] \right) + \\
     \tfrac{4}{3}  & \left(u[x-1,y,z;t] + u[x,y-1,z;t] + u[x,y,z-1;t] \right) + \\                     
     \tfrac{4}{3}  & \left(u[x+1,y,z;t] + u[x,y+1,z;t] + u[x,y,z+1;t] \right) + \\
    -\tfrac{1}{12}& \left(u[x+2,y,z;t] + u[x,y+2,z;t] + u[x,y,z+2;t] \right) \!\!\Big).
\end{align*}
This can now be turned into a \textsc{Patus} stencil specification almost trivially:

\begin{lstlisting}[language=stencil]
stencil wave(
  float grid U(0..x_max-1, 0..y_max-1, 0..z_max-1), 
  float param fMin = -1,
  float param fDX = 2 / (x_max-3),
  float param fDT_DX_sq = 0.25)
{
  // do one timestep within the stencil kernel
  iterate while t < 1;
  
  // define the region on which the stencil is evaluated
  domainsize = (2..x_max-3, 2..y_max-3, 2..z_max-3);
  
  // define the initial condition (how the data is initialized
  // before the computation)
  // note that 0 <= x < x_max, etc.
  initial {
    U[x,y,z; -1] = sinf(2*~$\pi$~*((x-1)*fDX+fMin)) *
      sinf(2*~$\pi$~*((y-1)*fDX+fMin)) * sinf(2*~$\pi$~*((z-1)*fDX+fMin));
    U[x, y, z; -1 :
      x==0 || y==0 || z==0 || x==x_max-1 || y==y_max-1 || z==z_max-1 ] = 0;
    U[x,y,z; 0] = U[x,y,z; -1];
    U[x,y,z; 1] = 0;
  }

  // define the actual stencil computation
  operation {
    float c1 = 2 - 15/2 * fDT_DX_sq;
    float c2 = 4/3 * fDT_DX_sq;
    float c3 = -1/12 * fDT_DX_sq;
    
    U[x,y,z; t+1] = c1 * U[x,y,z; t] - U[x,y,z; t-1] +
      c2 * (
        U[x+1,y,z; t] + U[x-1,y,z; t] +
        U[x,y+1,z; t] + U[x,y-1,z; t] +
        U[x,y,z+1; t] + U[x,y,z-1; t]
      ) +
      c3 * (
        U[x+2,y,z; t] + U[x-2,y,z; t] +
        U[x,y+2,z; t] + U[x,y-2,z; t] +
        U[x,y,z+2; t] + U[x,y,z-2; t]
      );
  }
}
\end{lstlisting}

\noindent From this stencil specification, \textsc{Patus} will generate
\begin{itemize}
  \item a C source code file implementing the stencil computation and
  \item source files from which a benchmarking harness can be built.
\end{itemize}
The benchmarking executable is then used by the auto-tuner, which determines a set of
architecture-specific parameters, for which the stencil achieves the best performance.


\subsubsection{The Specification Explained}

\begin{itemize}
  \item The \textbf{stencil} specification defines a stencil ``wave,'' which operates on a grid (called ``U'')
    of size $[$0, x\_max-1$] \times [$0, y\_max-1$] \times [$0,z\_max-1$]$.
    Note that the size parameters do not have to be defined in the stencil specification.
    Instead, they will appear as arguments to the generated stencil kernel function, and, in the
    benchmarking harness, as command line arguments.
    
    The remaining arguments, ``fMin,'' ``fDX,'' and ``fDT\_DX\_sq,'' are used for initializing the grid
    and performing the stencil computation. These parameters will also appear as arguments to the generated
    C function implementing the computation.

    Optionally, parameters can be initialized with default
    values (as shown in the listing above). This affects only the benchmarking harness and will fix the
    values which are passed to the generated stencil kernel.

  \item The \textbf{iterate while} statement defines the number of timesteps to be performed within one stencil kernel call.
    In the example, one timestep per kernel invocation will be performed. (Timesteps are counted from 0.)
    The statement can be omitted. Then, the number of timesteps defaults to 1.
    In the future we will also allow (reduction-based) convergence criteria (e.g., ``iterate as long as the residual is
    larger than some $\varepsilon$'').
    
  \item The \textbf{domainsize} defines the iteration space. While the total size of the grid extends from 0 to $\ast$\_max-1,
    the stencil is only applied to the points between 2 and $\ast$\_max-3 ($\ast \in \{$x, y, z$\}$), i.e., only
    to the \emph{interior} grid points.
    
  \item The \textbf{initial} block defines how the grid is initialized, or more mathematically: it defines the initial condition
    of the (discretized) PDE. The ``sinf'' function is actually a C function (single precision sine)
    which will be called (this is the behavior
    if there is a function which is not known to \textsc{Patus}).
    Note that \textsc{Patus} also defines the $\pi$ literal with the obvious meaning.\\
    The second statement of the initialization sets all the grid points for which x, y, z are 0 or $\ast$\_max-1
    (the condition after colon) to zero. I.e., you can use the set builder notation (with any logical and comparison operators you know from C/C++)
    to select certain grid points and initialize them.\\
    \textbf{initial} blocks are not mandatory; if no \textbf{initial} is provided, \textsc{Patus} will create an initialization
    routine anyway, initializing the data with arbitrary values. (This is to ensure the correct data placement on NUMA machines.)
    
  \item The \textbf{operation}, finally, defines the actual stencil computation.
    It can contain definitions of constants (as in the listing) or temporary values, and it can also contain more than
    one stencil expression.
  
  \item Optionally, the stencil specification can also contain a \textbf{boundaries} block, in which
    special treatment of boundary regions can be specified. Essentially, within the \textbf{boundaries} block,
    special stencils are defined which are applied to boundary regions. See section \ref{sec:stencilspec}.
\end{itemize}


\subsubsection{Building the Benchmarking Harness}

Once the stencil specification is written, \textsc{Patus} can be run to transform it into C code.
Assume the stencil specification was saved in the file \texttt{examples/ stencils/wave-1.stc}.
(This very stencil specification is actually there.)
In your shell, type
\cmd{cd examples/stencils\\
%  patus examples/stencils/wave-1.stc --outdir=out/wave-1
  \phantom{XXXX}patus wave-1.stc
}
This will generate the C code implementing the Wave stencil, using the default architecture,
and putting the generated files in the
\texttt{out} directory. (You can change the output directly by adding the \texttt{--outdir=$<$your-output-dir$>$}
to the \textsc{Patus} command line.)
Change into the output directory, \cmd{cd out}
and type \cmd{make} to build the benchmarking harness, which will be used for the auto-tuning process, and can also be
used for simple simulations. When the build completes successfully, there is an executable file,
\texttt{bench} (or \texttt{bench.exe} on Windows) in the directory.

If you try to start the benchmarking executable, you will see that it expects some command line arguments:
\example{\small\texttt{%
  \$  ./bench\\
  Wrong number of parameters. Syntax:\\
  ./bench <x\_max> <y\_max> <z\_max> <cb\_x> <cb\_y> <cb\_z> <chunk> <\_unroll\_p3>
}}

The $\ast$\_max correspond to the unspecified domain size variables in the stencil specification.
All the other arguments come from the code generator, and it is the auto-tuner's task to find the best values
for them.


\subsubsection{Auto-Tuning}

The Makefile (which was also used to build the benchmarking harness) defines a ``tune'' target, which
starts the auto-tuner. ``tune'' expects the domain size variables to be specified:
\cmd{make tune x\_max=64 y\_max=64 z\_max=64}
Type the above command in the shell. The auto-tuner will run the benchmarking executable a number of times,
varying the values of the arguments to determine, and it will terminate with a message like

\example{\small
  \texttt{Optimal parameter configuration found:}\\
  \texttt{64 64 64 62 20 8 2 0}\\

  \texttt{Timing information for the optimal run:}\\
  \texttt{0.9047935972598529}\\

  \texttt{Program output of the optimal run:}\\
  \texttt{Flops / stencil call:\phantom{XX}16}\\
  \texttt{Stencil computations:\phantom{XX}1116000}\\
  \texttt{Bytes transferred:\phantom{XXXXX}15728640}\\
  \texttt{Total Flops:\phantom{XXXXXXXXXXX}17856000}\\
  \texttt{Seconds elapsed:\phantom{XXXXXXX}0.000423}\\
  \texttt{Performance:\phantom{XXXXXXXXXXX}42.193517 GFlop/s}\\
  \texttt{Bandwidth utilization:\phantom{X}37.166590 GB/s}\\
  \texttt{1352692.000000}\\
  \texttt{Validation OK.}
}

\noindent Thus, according to the auto-tuner, on the machine the benchmark was run and for x\_max=y\_max=z\_max=64,
the best parameter combination is \cmd{cb\_x=62, cb\_y=20, cb\_z=8, chunk=2, \_unroll\_p3=0.}


\subsubsection{Simple Visualization}

You can run the benchmarking executable with the arguments provided by the auto-tuner,
\cmd{./bench 64 64 64 62 20 8 2 0}
If you add the flag \texttt{-o} to the command line,
\cmd{./bench 64 64 64 62 20 8 2 0 -o}
the program will take snapshots of the data grids before and after the execution of the stencil kernel
and write them to text files. If you have gnuplot installed on your system, you can run
\cmd{make plot} which will use gnuplot to create image files from the data text files.


\subsection{Writing Your Own Stencil Specifications}
\label{sec:stencilspec}

\subsubsection{Stencil Arguments}

\subsubsection{The ``operation''}

stencil offsets have to be compile-time constants

\begin{lstlisting}[language=stencil]
stencil wave(
  float grid U(0..x_max-1, 0..y_max-1, 0..z_max-1), 
  float param fDT_DX_sq = 0.25)
{
  // define the region on which the stencil is evaluated
  domainsize = (2..x_max-3, 2..y_max-3, 2..z_max-3);
  
  // define the actual stencil computation
  operation {
    float c[0..2] = {
      2 - 15/2 * fDT_DX_sq,
      4/3 * fDT_DX_sq,
      -1/12 * fDT_DX_sq
    };
    
    U[x,y,z; t+1] = c[0]*U[x,y,z; t] - U[x,y,z; t-1] +
      { i=-1..1, j=-1..1, k=-1..1, r=1..2 : i^2+j^2+k^2==1 } sum(
        c[r] * U[x+r*i,y+r*j,z+r*k; t]
      );
  }
}
\end{lstlisting}


\subsubsection{Number of Timesteps and Iteration Space}

\subsubsection{Boundary Conditions}

\subsubsection{Initial Condition}

set comprehensions/set-builder notation


\subsection{Integrating Into Your Own Application}

Two modes.

See the directory \texttt{examples/applications} for examples how to integrate \textsc{Patus}-generated code
into user code.


Create Fortran example