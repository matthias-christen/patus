% *************** Document style definitions ***************

% ******************************************************************
% This file defines the document design.
% Usually it is not necessary to edit this file, but you can change
% the design if you want.
% ******************************************************************

% *************** Load packages ***************

\usepackage{mathpazo} % math & rm
\linespread{1.05}        % Palatino needs more leading (space between lines)

\usepackage{graphicx}
\usepackage{epsfig}
\usepackage{amsmath}
\usepackage{amssymb}
\usepackage{amsthm}
\usepackage{booktabs}
\usepackage{stmaryrd}
\usepackage{url}
\usepackage{longtable}
\usepackage[figuresright]{rotating}
\usepackage{listings}
\usepackage{ebnf}
\usepackage[chapter]{algorithm}
\usepackage{algpseudocode}
%\usepackage{fixme}
\usepackage{mathabx}
\usepackage{multirow}
\usepackage{colortbl}
\usepackage{array}
\usepackage{tikz}
\usepackage{longtable}
\usepackage{framed}


% *************** Enable index generation ***************
\makeindex

% *************** Add reference to page number at which bibliography entry is cited ***************
\usepackage{citeref}
\renewcommand{\bibitempages}[1]{\newblock {\scriptsize [\mbox{cited at p.\ }#1]}}

% *************** Some colour definitions ***************
\usepackage{color}

\definecolor{greenyellow}   {cmyk}{0.15, 0   , 0.69, 0   }
\definecolor{yellow}        {cmyk}{0   , 0   , 1   , 0   }
\definecolor{goldenrod}     {cmyk}{0   , 0.10, 0.84, 0   }
\definecolor{dandelion}     {cmyk}{0   , 0.29, 0.84, 0   }
\definecolor{apricot}       {cmyk}{0   , 0.32, 0.52, 0   }
\definecolor{peach}         {cmyk}{0   , 0.50, 0.70, 0   }
\definecolor{melon}         {cmyk}{0   , 0.46, 0.50, 0   }
\definecolor{yelloworange}  {cmyk}{0   , 0.42, 1   , 0   }
\definecolor{orange}        {cmyk}{0   , 0.61, 0.87, 0   }
\definecolor{burntorange}   {cmyk}{0   , 0.51, 1   , 0   }
\definecolor{bittersweet}   {cmyk}{0   , 0.75, 1   , 0.24}
\definecolor{redorange}     {cmyk}{0   , 0.77, 0.87, 0   }
\definecolor{mahogany}      {cmyk}{0   , 0.85, 0.87, 0.35}
\definecolor{maroon}        {cmyk}{0   , 0.87, 0.68, 0.32}
\definecolor{brickred}      {cmyk}{0   , 0.89, 0.94, 0.28}
\definecolor{red}           {cmyk}{0   , 1   , 1   , 0   }
\definecolor{orangered}     {cmyk}{0   , 1   , 0.50, 0   }
\definecolor{rubinered}     {cmyk}{0   , 1   , 0.13, 0   }
\definecolor{wildstrawberry}{cmyk}{0   , 0.96, 0.39, 0   }
\definecolor{salmon}        {cmyk}{0   , 0.53, 0.38, 0   }
\definecolor{carnationpink} {cmyk}{0   , 0.63, 0   , 0   }
\definecolor{magenta}       {cmyk}{0   , 1   , 0   , 0   }
\definecolor{violetred}     {cmyk}{0   , 0.81, 0   , 0   }
\definecolor{rhodamine}     {cmyk}{0   , 0.82, 0   , 0   }
\definecolor{mulberry}      {cmyk}{0.34, 0.90, 0   , 0.02}
\definecolor{redviolet}     {cmyk}{0.07, 0.90, 0   , 0.34}
\definecolor{fuchsia}       {cmyk}{0.47, 0.91, 0   , 0.08}
\definecolor{lavender}      {cmyk}{0   , 0.48, 0   , 0   }
\definecolor{thistle}       {cmyk}{0.12, 0.59, 0   , 0   }
\definecolor{orchid}        {cmyk}{0.32, 0.64, 0   , 0   }
\definecolor{darkorchid}    {cmyk}{0.40, 0.80, 0.20, 0   }
\definecolor{purple}        {cmyk}{0.45, 0.86, 0   , 0   }
\definecolor{plum}          {cmyk}{0.50, 1   , 0   , 0   }
\definecolor{violet}        {cmyk}{0.79, 0.88, 0   , 0   }
\definecolor{royalpurple}   {cmyk}{0.75, 0.90, 0   , 0   }
\definecolor{blueviolet}    {cmyk}{0.86, 0.91, 0   , 0.04}
\definecolor{periwinkle}    {cmyk}{0.57, 0.55, 0   , 0   }
\definecolor{cadetblue}     {cmyk}{0.62, 0.57, 0.23, 0   }
\definecolor{cornflowerblue}{cmyk}{0.65, 0.13, 0   , 0   }
\definecolor{midnightblue}  {cmyk}{0.98, 0.13, 0   , 0.43}
\definecolor{darkblue}      {cmyk}{0.92, 0.67, 0   , 0.53}
\definecolor{navyblue}      {cmyk}{0.94, 0.54, 0   , 0   }
\definecolor{royalblue}     {cmyk}{1   , 0.50, 0   , 0   }
\definecolor{blue}          {cmyk}{1   , 1   , 0   , 0   }
\definecolor{cerulean}      {cmyk}{0.94, 0.11, 0   , 0   }
\definecolor{cyan}          {cmyk}{1   , 0   , 0   , 0   }
\definecolor{processblue}   {cmyk}{0.96, 0   , 0   , 0   }
\definecolor{skyblue}       {cmyk}{0.62, 0   , 0.12, 0   }
\definecolor{turquoise}     {cmyk}{0.85, 0   , 0.20, 0   }
\definecolor{tealblue}      {cmyk}{0.86, 0   , 0.34, 0.02}
\definecolor{aquamarine}    {cmyk}{0.82, 0   , 0.30, 0   }
\definecolor{bluegreen}     {cmyk}{0.85, 0   , 0.33, 0   }
\definecolor{emerald}       {cmyk}{1   , 0   , 0.50, 0   }
\definecolor{junglegreen}   {cmyk}{0.99, 0   , 0.52, 0   }
\definecolor{seagreen}      {cmyk}{0.69, 0   , 0.50, 0   }
\definecolor{green}         {cmyk}{1   , 0   , 1   , 0   }
\definecolor{forestgreen}   {cmyk}{0.91, 0   , 0.88, 0.12}
\definecolor{pinegreen}     {cmyk}{0.92, 0   , 0.59, 0.25}
\definecolor{limegreen}     {cmyk}{0.50, 0   , 1   , 0   }
\definecolor{yellowgreen}   {cmyk}{0.44, 0   , 0.74, 0   }
\definecolor{springgreen}   {cmyk}{0.26, 0   , 0.76, 0   }
\definecolor{olivegreen}    {cmyk}{0.64, 0   , 0.95, 0.40}
\definecolor{rawsienna}     {cmyk}{0   , 0.72, 1   , 0.45}
\definecolor{sepia}         {cmyk}{0   , 0.83, 1   , 0.70}
\definecolor{brown}         {cmyk}{0   , 0.81, 1   , 0.60}
\definecolor{tan}           {cmyk}{0.14, 0.42, 0.56, 0   }
\definecolor{gray}          {cmyk}{0   , 0   , 0   , 0.50}
\definecolor{lightgray}     {cmyk}{0   , 0   , 0   , 0.05}
\definecolor{black}         {cmyk}{0   , 0   , 0   , 1   }
\definecolor{white}{cmyk}{0   , 0   , 0   , 0   }%




% *************** Define our own pseudo language for lstlisting ***************
\lstdefinelanguage{pseudo}{
	morekeywords={procedure,function,if,else,for,all,in,to,downto,while,do,end},
	sensitive=true,
	morecomment=[l]{//},
	morecomment=[s]{/*}{*/},
	morestring=[b]"
}

\lstdefinelanguage{pseudo}{
	morekeywords={procedure,function,if,else,for,by,all,in,to,downto,while,do,end},
	sensitive=true,
	morecomment=[l]{//},
	morecomment=[s]{/*}{*/},
	morestring=[b]"
}

\lstdefinelanguage{strategy}{
  keywords={,strategy,stencil,for,by,schedule,subdomain,plane,point,in,parallel,temporary,alias,domain,auto,int,dim,},
  comment=[l]//,
  morecomment=[s]{/*}{*/}
}

\lstdefinelanguage{stencil}{
  keywords={,stencil,operation,boundary,filter,grid,double,float,const,param,domainsize,t_max,options,},
  comment=[l]//,
  morecomment=[s]{/*}{*/}
}

\lstset{
  language=pseudo,
  basicstyle=\ttfamily,
  frame=ltrb,
  framesep=0pt,
  backgroundcolor=\color{lightgray},
  rulecolor=\color{gray},
  basicstyle=\footnotesize,
  stringstyle=\ttfamily,
  commentstyle=\ttfamily\color{forestgreen},
  keywordstyle=\ttfamily\color{midnightblue}\bfseries,
  identifierstyle=\ttfamily, 
  tabsize=2,
  showstringspaces=false,
  escapechar=~,
  captionpos=b
}

% *************** Enable hyperlinks in PDF documents ***************
\ifpdf
    \pdfcompresslevel=9
        \usepackage[plainpages=false,pdfpagelabels,bookmarksnumbered,%
        colorlinks=true,%
        linkcolor=darkblue,%
        citecolor=darkblue,%
        filecolor=darkblue,%
        urlcolor=darkblue,%
        pdftex,%
        unicode]{hyperref} 
    \input supp-mis.tex
    \input supp-pdf.tex
    \pdfimageresolution=600
    \usepackage{thumbpdf} 
\else
    \usepackage{hyperref}
\fi

\usepackage{memhfixc}

% *************** Page layout ***************
\settypeblocksize{*}{32pc}{1.618}

\setlrmargins{*}{1.47in}{*}
\setulmargins{*}{*}{1.3}

\setheadfoot{\onelineskip}{2\onelineskip}
\setheaderspaces{*}{2\onelineskip}{*}

\def\baselinestretch{1.1}

\checkandfixthelayout

% *************** Chapter and section style ***************
\makechapterstyle{mychapterstyle}{%
    \renewcommand{\chapnamefont}{\LARGE\sffamily\bfseries}%
    \renewcommand{\chapnumfont}{\LARGE\sffamily\bfseries}%
    \renewcommand{\chaptitlefont}{\Huge\sffamily\bfseries}%
    \renewcommand{\printchaptertitle}[1]{%
        \chaptitlefont\hrule height 0.5pt \vspace{1em}%
        {##1}\vspace{1em}\hrule height 0.5pt%
        }%
    \renewcommand{\printchapternum}{%
        \chapnumfont\thechapter%
        }%
}

\chapterstyle{mychapterstyle}

\setsecheadstyle{\Large\sffamily\bfseries}
\setsubsecheadstyle{\large\sffamily\bfseries}
\setsubsubsecheadstyle{\normalfont\sffamily\bfseries}
\setparaheadstyle{\normalfont\sffamily}

\makeevenhead{headings}{\thepage}{}{\small\slshape\leftmark}
\makeoddhead{headings}{\small\slshape\rightmark}{}{\thepage}

% *************** Table of contents style ***************
\settocdepth{subsection}

\setsecnumdepth{subsection}
\maxsecnumdepth{subsection}
\settocdepth{subsection}
\maxtocdepth{subsection}

% ********** Commands for epigraphs **********
\setlength{\epigraphwidth}{0.57\textwidth}
\setlength{\epigraphrule}{0pt}
\setlength{\beforeepigraphskip}{1\baselineskip}
\setlength{\afterepigraphskip}{2\baselineskip}

\newcommand{\epitext}{\sffamily\itshape}
\newcommand{\epiauthor}{\sffamily\scshape ---~}
\newcommand{\epititle}{\sffamily\itshape}
\newcommand{\epidate}{\sffamily\scshape}
\newcommand{\episkip}{\medskip}

\newcommand{\myepigraph}[4]{%
	\epigraph{\epitext #1\episkip}{\epiauthor #2\\\epititle #3 \epidate(#4)}\noindent}

% *************** Algorithms ***************
\renewcommand{\algorithmicrequire}{\textbf{Input:}}
\renewcommand{\algorithmicensure}{\textbf{Output:}}

% *************** Theorems ***************
\newtheorem{theorem}{Theorem}[chapter]
\newtheorem{definition}{Definition}[chapter]

% *************** Other ***************
\renewcommand{\thefootnote}{\fnsymbol{footnote}}

\newcommand{\boldgreek}[1]{\mbox{\boldmath$#1$}}
\DeclareMathOperator*{\argmin}{arg\,min}


% *************** Examples ***************
\newcounter{example}[chapter]
\renewcommand\theexample{\thechapter.\arabic{example}} 
%\newcommand{\example}[2]{
%    \refstepcounter{example}%
%    \vspace{8pt}%
%    \noindent\fcolorbox{gray}{lightgray}{%
%        \begin{minipage}{0.985\textwidth}%
%	    \textbf{Example \theexample}: %
%	    \textit{#1}\hrule%
%	    \medskip%
%	    #2%
%       \end{minipage}%
%    }%
%    \vspace{8pt}%
%}

\makeatletter
\newcommand\framename{Example}
%\newcounter{framecnt}
%\setcounter{framecnt}{0}
\newcommand{\TitleFrame}[2]{%
    \fboxrule=\FrameRule
    \fboxsep=\FrameSep
    \vbox{\nobreak \vskip -0.7\FrameSep
        \rlap{\strut#1}\nobreak\nointerlineskip% left justified
        %\vskip 0.7\FrameSep
	    %\fontfamily{cmss}\fontseries{m}\fontshape{n}\selectfont
        \noindent\fcolorbox{gray}{lightgray}{#2}%
    }%
}
\newenvironment{example}[2][\FrameFirst@Lab\ (cont.)]{%
    \refstepcounter{example}%
    \def\FrameFirst@Lab{\textbf{\framename\ \theexample:\ }\textit{#2}}%
    \def\FrameCont@Lab{#1}%
    \def\FrameCommand##1{%
        \TitleFrame{\FrameFirst@Lab}{##1}}%
    \def\FirstFrameCommand##1{%
        \TitleFrame{\FrameFirst@Lab}{##1}}%
    \def\MidFrameCommand##1{%
        \TitleFrame{\FrameCont@Lab}{##1}}%
    \def\LastFrameCommand##1{%
        \TitleFrame{\FrameCont@Lab}{##1}}%
    \MakeFramed{\hsize\textwidth
    \advance\hsize -2\FrameRule
    \advance\hsize -2\FrameSep
    \FrameRestore}}%
   {\endMakeFramed}
\makeatother
% http://tex.stackexchange.com/questions/19099/framebox-and-minipage-causes-white-space-and-is-not-float


\newcommand{\cmdln}[3]{
	%\vspace{8mm}
	\noindent\texttt{--{#1}=}#2
	\begin{quote}#3\end{quote}
}


\newcolumntype{x}[1]{%
>{\centering\arraybackslash}p{#1}}%
% found here: http://texblog.wordpress.com/2008/05/07/fwd-equal-cell-width-right-and-centre-aligned-content/

\newcommand{\w}[1]{\textcolor{white}{\tiny #1}}%

\newcommand\diag[4]{%
  \multicolumn{1}{p{#2}|}{\hskip-\tabcolsep
  $\vcenter{\begin{tikzpicture}[baseline=0,anchor=south west,inner sep=#1]
  \path[use as bounding box] (0,0) rectangle (#2+2\tabcolsep,\baselineskip);
  \node[minimum width={#2+2\tabcolsep},minimum height=\baselineskip+\extrarowheight] (box) {};
  \draw (box.north west) -- (box.south east);
  \node[anchor=south west] at (box.south west) {#3};
  \node[anchor=north east] at (box.north east) {#4};
 \end{tikzpicture}}$\hskip-\tabcolsep}}
% from: http://tex.stackexchange.com/questions/17745/diagonal-lines-in-table-cell

\newcommand\bdiag[4]{%
  \multicolumn{1}{p{#2}|}{\hskip-\tabcolsep
  $\vcenter{\begin{tikzpicture}[baseline=0,anchor=south west,inner sep=#1]
  \path[use as bounding box] (0,0) rectangle (#2+2\tabcolsep,\baselineskip);
  \node[minimum width={#2+2\tabcolsep},minimum height=\baselineskip+\extrarowheight] (box) {};
  \draw (box.north east) -- (box.south west);
  \node[anchor=north west] at (box.north west) {#3};
  \node[anchor=south east] at (box.south east) {#4};
 \end{tikzpicture}}$\hskip-\tabcolsep}}
 

\newcommand{\fixme}[1]{\textbf{\color{red}!!! #1 !!!}}


% *************** End of document style definition ***************
