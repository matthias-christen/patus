\chapter{\textsc{Patus} Usage}
\label{sec:cmdline}

\section{Code Generation}

The \textsc{Patus} code generator is invoked by the following command:
\vspace{4mm}
\begin{lstlisting}
	java ~\texttt{-}~jar patus.jar codegen
		~\texttt{--}~stencil=~\textit{$<$Stencil File$>$}~  ~\texttt{--}~strategy=~\textit{$<$Strategy File$>$}~
		~\texttt{--}~architecture=~\textit{$<$Architecture Description File$>$}~,~\textit{$<$Hardware Name$>$}~
		~[ \texttt{--}~outdir=~\textit{$<$Output Directory$>$} ] [\texttt{--}~generate=~\textit{$<$Target$>$} ]~
		~[ \texttt{--}~kernel-file=~\textit{$<$Kernel Output File Name$>$} ]~
		~[ \texttt{--}~compatibility=~\{~ C ~$|$~ Fortran ~\} ]~
		~[ \texttt{--}~unroll=~\textit{$<$UnrollFactor1$>$}~,~\textit{$<$UnrollFactor2$>$}~,...~ ]~
		~[ \texttt{--}~use-native-simd-datatypes=~\{~ yes ~$|$~ no ~\} ]~
		~[ \texttt{--}~create-validation=~\{~ yes ~$|$~ no ~\} ]~
		~[ \texttt{--}~validation-tolerance=~\textit{$<$Tolerance$>$} ]~
		~[ \texttt{--}~debug=~\textit{$<$DebugOption1$>$}~,~[\textit{$<$DebugOption2$>$}~,~[~...,~[\textit{$<$DebugOptionN$>$} ]~...~]]~
\end{lstlisting}
\vspace{4mm}

\cmdln{stencil}{\textit{$<$Stencil File$>$}}{Specifies the stencil specification file for which to generate code.}
\cmdln{strategy}{\textit{$<$Strategy File$>$}}{The Strategy file describing the parallelization/optimization strategy.}
\cmdln{architecture}{\textit{$<$Architecture Description File$>$},\textit{$<$Hardware Name$>$}}{
	The architecture description file and the name of the selected
	architecture (as specified in the \texttt{name} attribute of the
	\texttt{architectureType} element).
}
\cmdln{outdir}{\textit{$<$Output Directory$>$}}{
	The output directory into which the generated files will be written.
	Optional; if not specified the generated files will be created in the
	current directory.
}
\cmdln{generate}{\textit{$<$Target$>$}}{
	The target that will be generated. \textit{$<$Target$>$} can be one of:
	\begin{itemize}
		\item \texttt{benchmark}\\
				This will generate a full benchmark harness.
				This is the default setting.
		\item \texttt{kernel}\\
				This will only generate the kernel file.
	\end{itemize}
}
\cmdln{kernel-file}{\textit{$<$Kernel Output File Name$>$}}{
	Specifies the name of the C source file to which the generated kernel
	is written. The suffix is appended or replaced from the definition
	in the hardware architecture description.\\
	Defaults to \texttt{kernel.c}.
}
\cmdln{compatibility}{\{\texttt{C} $|$ \texttt{Fortran}\}}{
	Selects whether the generated code has to be compatible with Fortran
	%(creates pointer-only input types to the kernel selection function).\\
	(Omits the double pointer output type in the kernel declaration; therefore
	multiple time steps are not supported.)
	Defaults to \texttt{C}.
}
\cmdln{unroll}{\textit{$<$UnrollFactor1$>$},\textit{$<$UnrollFactor2$>$},...}{
	A list of unrolling factors applied to the inner most loop nest
	containing the stencil computation. The unrolling factors are applied to all the dimensions.
}
\cmdln{use-native-simd-datatypes}{\{\texttt{yes} $|$ \texttt{no}\}}{
	Specifies whether the native SSE data type is to be used in the kernel
	signature. If set to \texttt{yes}, this also requires that the fields are padded correctly
	in unit stride direction.\\
	Defaults to \texttt{no}.
}
\cmdln{create-validation}{\{\texttt{yes} $|$ \texttt{no}\}}{
	Specifies whether to create code that will validate the result.
	If \textit{$<$Target$>$} is not \texttt{benchmark}, this option will be ignored.\\
	Defaults to \texttt{yes}.
}
\cmdln{validation-tolerance}{\textit{$<$Tolerance$>$}}{
	Sets the tolerance for the relative error in the validation.
	This option is only relevant if validation code is generated
	(\texttt{--create-validation=yes}).\\
	Defaults to \texttt{yes}.
}
\cmdln{debug}{\textit{$<$DebugOption1$>$},[\textit{$<$DebugOption2$>$},[...,[\textit{$<$DebugOptionN$>$}]...]]}{
	Specifies debug options (as a comma-separated list) that will influence
	the code generator.
	Valid debug options (for \textit{$<$DebugOptioni$>$}, $i=1,\dots,N$) are:
	\begin{itemize}
		\item \texttt{print-stencil-indices}\\
			This will insert a \texttt{printf} statement for
			every stencil calculation with the index
			into the grid array at which the result
			is written.
		\item \texttt{print-validation-errors}\\
			Prints all values if the validation fails.
			The option is ignored if no validation code
			is generated.
	\end{itemize}
}


\section{Auto-Tuning}
\label{sec:appendix-autotuning}

The \textsc{Patus} auto-tuner is invoked on the command line like so:

\vspace{4mm}
\begin{lstlisting}
	java ~\texttt{-}~jar patus.jar autotune
		~\textit{$<$Executable Filename$>$}~
		~\textit{$<$Param1 $>$}~ ~\textit{$<$Param2 $>$}~ ... ~\textit{$<$ParamN $>$}~
		~[ \textit{$<$Constraint1$>$}~ ~\textit{$<$Constraint2$>$}~ ... ~\textit{$<$ConstraintM$>$} ]~
		~[ \texttt{-}m\textit{$<$Method$>$} ]~
\end{lstlisting}
\vspace{4mm}

\textit{$<$Executable Filename$>$} is the path to the file name of the benchmark executable.
The benchmark executable must expose the tunable parameters as command line parameters. The \textsc{Patus}
auto-tuner only generates numerical parameter values. If the benchmark executable requires strings, these
must be mapped from numerical values internally in the benchmarking program.

The parameters \textit{ParamI}, $I=1,\dots,N$, define integer parameter ranges and have the following syntax:
\begin{quote}
	\textit{$<$StartValue$>$}\texttt{:}[[\texttt{*}]\textit{$<$Step$>$}\texttt{:}]\textit{$<$EndValue$>$}[\texttt{!}]
\end{quote}
or
\begin{quote}
	\textit{$<$Value1$>$}[\texttt{,}\textit{$<$Value2$>$}[\texttt{,}\textit{$<$Value3$>$}...]][\texttt{!}]
\end{quote}
The first version, when no asterisk \texttt{*} in the \textit{$<$Step$>$} is specified, enumerates all values in
\[
	\{ a := \text{\textit{$<$StartValue$>$}} + k \cdot \text{\textit{$<$Step$>$}} : k \in \mathbb{N}_0 \text{ and } a \leq \text{\textit{$<$EndValue$>$}} \}.
\]
If no \textit{$<$Step$>$} is given, it defaults to $1$.
If there is an asterisk \texttt{*} in front of the \textit{$<$Step$>$}, the values enumerated are
\[
	\{ a := \text{\textit{$<$StartValue$>$}} \cdot \text{\textit{$<$Step$>$}}^k : k \in \mathbb{N}_0 \text{ and } a \leq \text{\textit{$<$EndValue$>$}} \}.
\]
In the second version, all the comma-separated values \textit{$<$Value1$>$}, \textit{$<$Value2$>$}, \dots are enumerated.

If the optional \texttt{!} is appended to the value  range specification,
each of the specified values is guaranteed to be used, i.e., an exhaustive search is used for the corresponding parameter.

\begin{example}{Specifying parameter ranges.}
	\noindent $1:10$\\
	\phantom{XXXX}enumerates all the integer numbers between and including $1$ and $10$.
	\bigskip
	
	\noindent $2:2:41$\\
	\phantom{XXXX}enumerates the integers $2,4,6,\dots,38,40$.
	\bigskip
	
	\noindent $1:*2:128$\\
	\phantom{XXXX}enumerates some powers of $2$, namely $1,2,4,8,16,32,64,128$.
\end{example}

Optional  constraints  can  be  specified  to restrict the parameter values. The syntax for the constraints is
\begin{quote}
	\texttt{C}\textit{$<$ComparisonExpression$>$}
\end{quote}
where \textit{$<$ComparisonExpression$>$} is an expression which can contain arithmetic operators \texttt{+}, \texttt{-}, \texttt{*}, and \texttt{/},
as well as comparison operators \texttt{<}, \texttt{<=}, \texttt{==}, \texttt{>=}, \texttt{>}, \texttt{!=},
and variables \texttt{\$}$1$, \dots, \texttt{\$}$N$, which correspond to the parameters \textit{$<$Param1 $>$}, \dots \textit{$<$ParamN $>$}.

\begin{example}{Constraints examples.}
	\noindent $C\$2<=\$1$\\
	\phantom{XXXX}forces the second parameter to be less or equal to the first.
	\bigskip
	
	\noindent $C(\$2+\$1-1)/\$1>=24$\\
	\phantom{XXXX}forces $\tfrac{\$2+\$1-1}{\$1} = \left\lceil \tfrac{\$2}{\$1} \right\rceil \geq 24$.
\end{example}

The \textsc{Patus} auto-tuner supports a range of search methods. The method can be selected by \texttt{-m}\textit{$<$Method$>$}
where \textit{$<$Method$>$} is one of
\begin{itemize}
	\item \texttt{ch.unibas.cs.hpwc.patus.autotuner.DiRectOptimizer}\\ DIRECT method
	\item \texttt{ch.unibas.cs.hpwc.patus.autotuner.ExhaustiveSearchOptimizer}\\ exhaustive search
	\item \texttt{ch.unibas.cs.hpwc.patus.autotuner.GeneralCombinedEliminationOptimizer}\\ general combined elimination
	\item \texttt{ch.unibas.cs.hpwc.patus.autotuner.GreedyOptimizer}\\ greedy search
	\item \texttt{ch.unibas.cs.hpwc.patus.autotuner.HookeJeevesOptimizer}\\ Hooke-Jeeves algorithm
	\item \texttt{ch.unibas.cs.hpwc.patus.autotuner.MetaHeuristicOptimizer}\\ genetic algorithm
	\item \texttt{ch.unibas.cs.hpwc.patus.autotuner.RandomSearchOptimizer}\\ draws $500$ random samples
	\item \texttt{ch.unibas.cs.hpwc.patus.autotuner.SimplexSearchOptimizer}\\ simplex search (aka Nelder-Mead method)
\end{itemize}
These are Java class paths to \texttt{IOptimizer} implementations; this allows to extend range of methods easily.
Please refer to Chapter \ref{sec:autotuning} for a discussion and comparison of the methods.
If no method is specified, the greedy algorithm is used by+- default.